\documentclass[a4paper, 11pt, hidelinks]{article}
\usepackage{bookmark}
\usepackage[utf8]{inputenc} 
\usepackage[T1]{fontenc}
\usepackage{lmodern}
\usepackage{graphicx}
\usepackage[french]{babel}
\usepackage{geometry}
\usepackage{eucal}
\usepackage{caption}
\usepackage{float}
\usepackage{url}
\usepackage{amsmath}
\usepackage{amssymb}
\usepackage{color}
\usepackage{hyperref}
\usepackage{cancel}
\usepackage{tikz}
\usepackage{mathrsfs}  
\usepackage{esvect}


\geometry{hmargin=2cm,vmargin=1.5cm}

\tikzset{
  treenode/.style = {shape=rectangle, rounded corners,
                     draw, align=center,
                     top color=white, bottom color=blue!5},
  root/.style     = {treenode, font=\Large, bottom color=red!10},
  env/.style      = {treenode, font=\ttfamily\normalsize},
  dummy/.style    = {circle,draw}
}

\newcommand{\prp}{\large \textbf{Proposition :} \large}

\newcommand{\tm}{\large \textbf{Théoreme :} \large}

\newcommand{\ex}{\textcolor{green}{Exemple :} }

\newcommand{\dm}{\textcolor{red}{\textbf{Démo :} } }

\newcommand{\de}{\large \textbf{Définition} \large }

\newcommand{\rmq}{\textbf{Remarque :} }

\newcommand{\bs}{\bigskip}

\newcommand{\voca}{\textcolor{blue}{\textbf{Vocabulaire} } }

\newcommand{\lem}{\textcolor{red}{\textbf{Lemme :} } }

\newcommand{\trinom}[3]{\begin{pmatrix}
    #1 \\
    #2 \\
    #3
\end{pmatrix}}

\newcommand{\quadrinom}[4]{\begin{pmatrix}
    #1 \\
    #2 \\
    #3 \\
    #4 \\
\end{pmatrix}}

\newcommand{\pentanom}[5]{\begin{pmatrix}
    #1 \\
    #2 \\
    #3 \\
    #4 \\
    #5
\end{pmatrix}}

\newcommand{\hexanom}[6]{\begin{pmatrix}
    #1 \\
    #2 \\
    #3 \\
    #4 \\
    #5 \\
    #6 
\end{pmatrix}}

\newcommand{\serie}[2]{\displaystyle\sum_{#1 =0}^{+\infty} #2_{#1} }

\newcommand{\tend}{\underset{n \to + \infty}{\longrightarrow} }

\newcommand{\Lra}{\Leftrightarrow}

\newcommand{\lra}{\leftrightarrow}

\newcommand{\Ra}{\Rightarrow}

\newcommand{\ra}{\rightarrow}

\newcommand{\la}{\leftarrow}

\newcommand{\La}{\Leftarrow}

\newcommand{\dsum}[2]{\displaystyle\sum_{#1}^{#2} }

\newcommand{\dint}[2]{\displaystyle\int_{#1}^{#2} }

\newcommand{\ntend}{\underset{n \to + \infty}{\not \longrightarrow} }

\newenvironment{lmatrix}{$ \left|\begin{array}{l} }{\end{array}\right.$}

\newcommand{\img}[4]{\begin{figure}[!ht]
    \centering
    \includegraphics[scale=#1 ]{#2}
    \caption{#3}
    \label{#4}
    \end{figure} }    
\begin{document}

\newcommand{\grad}[1]{\vv{grad}#1}


\title{Maths}
\author{Schobert Néo}

\maketitle

\tableofcontents

\newpage 



\section{Convergence simple / convergence uniforme}

\subsection{8 Novembre}

\subsection{Questions}

\begin{itemize}
    \item Qu'est-ce que la convergence simple ?
    \item Que dire de $g,h : A\subset D \to \mathbb{K}$ quand $(f_n)$ converge simplement (CVS) vers $g$ et $h$.
    \item Qu'est-ce que la limite simple (parler d'unicité)
    \item Propriétés qui se transmettent par convergence simple (2 trucs)
    \item Citer 2 propriétés qui ne se transmettent pas (citer 2 exemples)
    \item Quelle variable arrive en premier ? $x$ ou $n$ ?
    \item Parler de convergence simple / uniforme \textbf{sur un ensemble}
    \item Différence entre CVS et CVU.
    \item Quelle variable arrive en premier ? $x$ ou $n$ ?
    \item Propositions liées à la CVU. (restriction / CVU $\Rightarrow CVS$ / $||||_{+\infty}$ / unicité)
    \item Qu'est-ce que la limite uniforme.
    \item Transfert de la continuité en CVU. (re-démontrer)
    \item Notation $\delta_n(x)$ et comment l'utiliser ?
    \item Méthode pour montrer une non CVU.
    \item Rappeler notion de convergence uniforme / simple sur série d'application.
    \item Convergence simple de $\sum f_n$ sur $A$ (A définir)
    \item Grossière divergence en convergence de série d'application. (redémontrer)
    \item Caractérisation de la convergence uniforme de $\sum f_n$ sur $A$.
    \item Transfert de la continuité en CVU de séries d'applications.
    \item Quand dispose-t-on de $(R_n)$ et $S$ ?
    \item Définir convergence absolue et convergence normale d'une série d'application.
    \item Lien entre les $4$ convergences.
\end{itemize}




\subsubsection{Remarques}
\begin{itemize}
    \item Définir racine $n$-ième de $x \in \mathbb{R}$
\end{itemize}



\subsection{Session exercice 8:}


\subsubsection{Exercice 21 (KDR)}

Pour $x\in \mathbb{R}^*_+$, on pose $S(x)=\displaystyle\sum_{n=0}^{+\infty} \frac{(-1)^n}{x+n}$

\begin{enumerate}
    \item Mq $S$ est définie et continue sur $\mathbb{R}^*_+$.
    
    
    Pour la définition, montrer que $S(x)$ existe.

    Penser au CSSA ($(\frac{1}{x+n})$ est décroissante et tend vers $0$)

    Ensuite, passer par la convergence uniforme puis par le théorème de transfert de continuité.
    \item Soit $(x,y)\in \mathbb{R}^*_+ | x<y$.
    
    Etudier $S(y)-S(x)$, puis repenser au CSSA et au fait que la somme est du signe de son premier terme dans le CSSA.
    ($T(x)=\displaystyle\sum_{n=0}^{+ \infty} (-1)^n \frac{1}{x(n+1)}$ est du signe de son premier terme.)
    \item Le but ici est de vérifier une équation fonctionnelle pour $S(x)$.
    
    On calcul alors $S(x+1)$, que l'on exprime en fonction de $S(x)$. Par changement de variable, on obtient :

    $\forall x \in \mathbb{R}^*_+, S(x+1)=\frac{1}{x} - S(x)$

    $S$ est $C^0$ en 1. On étudie alors les limites: $S(x)=\frac{1}{x} - S(x+1)= \frac{1}{x} + o(\frac{1}{x}) \sim \frac{1}{x}$.
    
    Donc $S(x) \sim_{0^+} \frac{1}{x}$

    Pour l'équivalent en $+\infty$, il faut avoir l'idée \textbf{d'encadrer $S(x)$}. Pour cela, utiliser la relation fonctionnelle et la monotonie de S en $x$ et en $x-1$.
\end{enumerate}


\subsubsection{Exercice 13}

Pour $n \in \mathbb{N}$ et $x\in \mathbb{R}$, on pose: $f_n(x)=\frac{sin(nx)}{n!}$

Montrer que la série de fonctions $\sum f_n$ converge uniformément sur $\mathbb{R}$ puis expliciter sa somme.

Ici, plutôt que de montrer la convergence uniforme en utilisant la convergence simple puis le reste $R_n$ (qui doit tendre vers $0$),
on préfère utiliser la convergence normale.

$\forall x \in \mathbb{R}$, $|f_n(x)|\leq \frac{1}{n!}$.

Donc comme la borne supérieur et le plus petit des majorants, $||f_n||_{\infty}\leq \frac{1}{n!}$

Par comparaison de SATP, $\sum ||f_n||_{\infty}$ converge.

Donc on a convergence normale donc uniforme.

Elle converge donc simplement. On dispose alors de la somme $S(x)$.

On y étudie la limite sachant que la fonction $Im$ est $\mathbb{R}$ linéaire et $dim_{\mathbb{R}}(\mathbb{C})=2< +\infty$ donc $Im$ est $C^0$ sur $\mathbb{C}$.




\subsubsection{Exercice 7}

Soit $(f_n)$ une suite de fonctions polynômes qui converge uniformément sur $\mathbb{R}$ vers $f$.

Montrer que $f$ est une fonction polynôme.


Celui-ci est pas évident.

Il faut en fait repartir des définitions.

$f_n$ est une fonction polynôme donc $\exists P_n \in \mathbb{R}[X] | \forall t \in \mathbb{R}, f_n(t)=P_n(t)$.

On a alors $||f_n -f ||_{\infty} \tend 0$

On remarque alors que toute fonction polynôme bornée sur $\mathbb{R}$ est constante.


On s'intéresse alors aux "tranches de Cauchy".

C'est à dire, à $||f_{n+p}-f_n||_{\infty}$.

On montre que cette expression est bornée.

On a alors $\exists c_{n,p} \in \mathbb{R} | \forall t \in \mathbb{R}, f_{n+p}(t) -f_n(t)=c_{n,p}$

$c_{n,p}= f_{n,p}(12)-f_n(12) \underset{p \to + \infty}{\longrightarrow} f(12) - f_n(12)$

On a alors avec $c_n$ la limite quand $p \to +\infty$ de $c_{n,p}$,

$f(t)-f_n(t)=c_n$

Donc $f(t)=f_N(t)+c_n$ $f_N$ est une fonction pôlynomiale donc $f$ est une fonction polynômiale.




\subsubsection{Exercice 16}

On pose $S(x)=\displaystyle\sum_{n=1}^{+\infty} \underbrace{n^xe^{-nx}}_{f_n(x)}$. Montrer que $S$ est continue sur $]0,+\infty [$

Soit $x \in \mathbb{R}^*_+$,

$f_n(x)=e^{-(n-ln(x))x}= o(\frac{1}{n^2})$.

Donc $\sum f_n(x)$ converge car $2>1$

Donc $S(x)$ existe. Donc $S$ est définie sur $\mathbb{R}^*_+$.

On ne peut pas utiliser la convergence normale ici car on s'aperçoit en faisant le tableau de signe,

que $||f_n||_{\infty}$=1

On se place alors sur un segment où ca marche et ok par convergence normale... 



\subsection{12 Novembre}


\subsubsection{Questions}

\begin{itemize}
    \item Définir le rayon de convergence.
    \item Lien $r$ / borne.
    \item Rayon de convergence de la suite nulle.
    \item Rayon de convergence d'une suite constante non nulle.
    \item Si la suite $(a_n)$ est bornée, alors $R_a$...
    \item Si la suite est convergente de limite non nulle alors $R_a$...
    \item Si la suite tend vers zéro, alors $R_a$...
    \item Si $|a_n| \leq |b_n|$, alors $R_a$...$R_b$.
    \item Si $a_n= \mathcal{O}(b_n)$ alors $R_a$...$R_b$.
    \item Si $a_n \sim b_n$ alors $R_a$...$R_b$.
    \item Lien entre $R_{a+b}$ et $R_a$ et $R_b$.
    \item Lien entre $R_{\alpha a}$ et $R_a$.
    \item Suite $D(a)$ et $I(a)$. (suite dérivée et primitive)
    \item Lien suite série pour $R_a$.
    \item Règle d'Alembert.
    \item Rayon de convergence de la suite géométrique $(a^n)$
    \item Pour montrer $R_a \leq R_b$,
    
    On prend $r\in [0,R_a[$ puis on montrer que la suite $(b_nr^n)$ est bornée.
    \item $\forall h \in \mathbb{R}^*_+, \forall p \in \mathbb{N}, (r+h)^{p+1} \leq (p+1)r^ph$
    \item Cas de la fraction rationnelle \textbf{non nulle}.
    \item Cas du produit de convolution.
    \item Qu'est-ce qu'une série entière de la variable complexe.
    \item Rayon de convergence d'une série entière.
    \item Définition ensemble de convergence de $\sum a_nz^n$, disque ouvert de $\sum a_nz^n$ et cercle d'incertitude de $\sum a_nz^n$.
    \item Définition somme de la série entière.
    \item Que peut-on dire du disque fermé $D_r$. Et que ne peut-on pas dire sur le disque ouvert de convergence $D_a$ vis-à-vis de la convergence uniforme.
    \item Série entière produit des séries entières $\sum a_n z^n$ et $\sum b_n z^n$.
    \item $z\in \mathbb{C}$, $\in \mathbb{N}^*$, Rayon de convergence des séries $\dsum{n=0}{+\infty} z^n$, $\dsum{n=p}{+ \infty} n(n-1)...(n-(p-1))z^{n-p}$, $\dsum{n=0}{+ \infty} \binom{p}{n+p}z^n$
    et somme de ces séries. Moyen mnémotechnique pour le $2$. (dérivation)
    \item cas des fonctions trigonométriques / exponentielles et trigonométriques hyperboliques.
    \item Définition du cos d'un complexe... Revoir l'histoire de la définition des fonctions cos et sin.
    \item Tout pareil pour les séries entières de la variable réelle.
    \item Dernier théorème.
\end{itemize}




\subsubsection{Remarque}



\section{Integrale généralisée}

\subsection{15 Novembre}


\subsubsection{Questions}

\begin{itemize}
    \item Notion d'intégrale généralisée
    \item Dans le cas d'une fonction positive, son intégrale est croissante. Sa convergence équivaut alors à sa majoration. (comme les SATPs)
    \item Parties réelles et imaginaires stables par intégration.
    \item Croissance de l'intégrale
    \item Faire bien \textbf{attention aux convergences}
    \item Que ne faut-il pas écrire quand l'intégrale de $f+g$ converge et l'intégrale de $f$ et l'intégrale de $g$ divergent.
    \item Définir le reste de l'intégrale et les conditions de dérivabilité.
    \item Définir intégrale généralisée de $f$ sur $[a,b[$
    \item Dans le cas d'une fonction positive, son intégrale est croissante. Sa convergence équivaut alors à sa majoration. (comme les SATPs) (cas $[a,b[$)
    \item Une application continue par morceaux sur un \textbf{segment} est bornée.
    \item Condition pour avoir l'intégrale de $f$ convergente sur $[a,b[$ quand $f$ est continue par morceaux.
    \item Valeur de $arctanh$ en fonction de $ln$
    \item Quand on fait les calculs, on utilise la notation $F(x)=\displaystyle\int_a^x f$ 
    \item Tout pareil sur $]a,b]$
    \item Notion d'intégrale généralisée sur $]a,b[$
    \item Généralement, on prolonge par continuité la fonction.
    
    Exemple : $g(t)=ln(t)ln(1-t)$. 
    
    $g$ est continue sur $]0,1[$.

    On pose $g(0)=g(1)=0$.

    On a alors $g$ est continue sur $[0,1]$

    $g(t)\sim_{0^+} -tln(t)$     $g(t)\sim_{1^-} (t-1)ln(1-t)$

    Alors $\displaystyle\int_0^1 g$ converge.
    \item Dans le changement de variable MPSI, on pose $t=\varphi (u)$. Il faut alors seulement $\varphi \in \mathscr{C}^1$
    \item Dans le nouveau changement de variable, cas des intégrales impropres, il faut $\varphi : ]a,b[ \to ]\alpha,\beta[$ une bijection strictement monotone
    de classe $\mathscr{C}^1$ de $]\alpha,\beta[$ sur $]a,b[$.
    
    On a alors si $f$ est continue sur $]a,b[$, $\displaystyle\int_a^b f$ et $\displaystyle\int_\alpha^\beta (f \circ \varphi)\varphi '$ sont de même nature.

    Dans le cas où ça converge, $\displaystyle\int_a^b f(x)dx=\displaystyle\int_\alpha^\beta f(\varphi(t))|\varphi '(t)|dt$
    \item Dans l'intégration par partie MPSI, on doit avoir la fonction qu'on primitive qui est $\mathscr{C}^0$ et la fonction qu'on dérive qui est $\mathscr{C}^1$.
    \item Dans le nouveau théorème d'intégration par partie, cas des intégrales impropres, il faut en plus que, pour $\displaystyle\int_a^b u'v$,
    
    $uv$ admette une limite en $a$ et en $b$
    \item Définition d'une intégrale absolument convergente. D'une intégrale semi-convergente.
    \item Comparaison de deux fonctions continues par morceaux $0\leq \varphi \leq \psi$
    
    La convergence de l'intégrale de $\varphi$ se déduit alors de celle de $\psi$.

    La divergence de l'intégrale de $\psi$ se déduit alors de celle de $\varphi$.
    \item Conditions pour appliquer l'inégalité triangulaire à l'intégrale.
\end{itemize}




\subsubsection{Remarque}





\subsection{Session Exercice 9 :}


\subsubsection{Exercice 11}

$a_0=1$, $a_{n+1}=arctan(a_n)$


$f(x)=arctan(x)$

$g(x)=f(x)-x$

Avec un tableau de signe, $f$ est croissante.

puis $g'(x)=\frac{1}{1+x^2}-1=\frac{-x^2}{1+x^2}\leq 0$

$g(0)=0$ et $g$ décroissante donc :

$a_{n+1}-a_n=g(a_n)\leq 0$

$(a_n)$ décroissante et minorée par $0$ donc $(a_n)$ converge.

Posons $\ell=lim a_n$, $\ell \in \mathbb{R}^+$

$a_{n+1}=f(a_n)$ et $f'$ $\mathscr{C}^0$ sur $\mathbb{R}^+$ et PPL: $\ell = f(\ell)$

i.e $g(l)=0$ puis $\ell=0$

$(a_n)$ converge donc $R_a\geq 1$

$arctan(u)=u-\frac{u^3}{3} + o(u^3)$

$a_{n+1}=a_n-\frac{a_n^3}{3} + o(a_n^3)$     $lim a_n =0$

$a_{n+1}=a_n(1-\frac{a_n^2}{3} + o (a_n^2))$


on calcul alors $a_{n+1}^{-2}$ (l'idée viens du carré dans les parenthèses)

On montre ainsi que $lim a_{n+1}^{-2} -a_n^{-2}=\frac{2}{3}$

Ainsi, $a_n^{-2} \sim \frac{2n}{3}$

$a_n \sim \sqrt{\frac{3}{2n}}=\sqrt{\frac{3}{2}} \frac{1}{n^{\frac{1}{2}}}$

On montre alors que $R_a =1$ en étudiant $\frac{|b_{n+1}|}{|b_n|}$ avec $b_n=\frac{K}{n^{\frac{1}{2}}}$

On en déduit alors $E_a$ en étudiant le comportement aux limites.



\subsubsection{Exercice 1}

5)

$a_n=\frac{ch(n)}{n}$.

On a alors $R_a=R_{(na_n)}$

$na_n \sim ch(n) \sim \frac{e^n}{2}$

$R_a=R_{(e^n)}=\frac{1}{|e|}=\frac{1}{e}$



\subsubsection{Exercice 9}



\subsection{18 Novembre}

\begin{itemize}
    \item Notion d'intégrabilité.
    \item Ensemble $L^1(I,\mathbb{K})$
    \item Caractérisation de l'intégrabilité 
    \item Exemple de référence d'applications intégrables.
    \item Quelles sont les conditions pour écrire $\int_{I} f$
    \item intégrable $\Rightarrow$ intégrale généralisée converge. La réciproque est fausse.
    \item Toute intégrale semi-convergente converge mais la fonctions qui lui est associée n'est pas intégrable.
    \item Conditions pour avoir $\int_{I} \varphi \Rightarrow \varphi =0_{\mathbb{R}^I}$ (4 conditions)
\end{itemize}





\subsection{19 Novembre}


\subsubsection{Questions}


\begin{itemize}
    \item Si c'est intégrable, les bornes sont négligeables.
\end{itemize}


\subsubsection{Remarques}

\begin{itemize}
    \item 
\end{itemize}



\subsection{22 Novembre}




\subsubsection{Questions}


\begin{itemize}
    \item Rappeler le théorème de Riemann-Lebesgue
    \item Rappeler le théorème de la limite de la dérivée.
    \item Majoration de $|fg|$
    \item Théoreme de convergence dominée
\end{itemize}


\subsubsection{Remarques}






\end{document}