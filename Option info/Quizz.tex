\documentclass[a4paper, 11pt, hidelinks]{article}
\usepackage{bookmark}
\usepackage[utf8]{inputenc} 
\usepackage[T1]{fontenc}
\usepackage{lmodern}
\usepackage{graphicx}
\usepackage[french]{babel}
\usepackage{geometry}
\usepackage{eucal}
\usepackage{caption}
\usepackage{float}
\usepackage{url}
\usepackage{amsmath}
\usepackage{amssymb}
\usepackage{color}
\usepackage{hyperref}
\usepackage{cancel}
\usepackage{tikz}
\usepackage{mathrsfs}  
\usepackage{esvect}


\geometry{hmargin=2cm,vmargin=1.5cm}

\tikzset{
  treenode/.style = {shape=rectangle, rounded corners,
                     draw, align=center,
                     top color=white, bottom color=blue!5},
  root/.style     = {treenode, font=\Large, bottom color=red!10},
  env/.style      = {treenode, font=\ttfamily\normalsize},
  dummy/.style    = {circle,draw}
}

\newcommand{\prp}{\large \textbf{Proposition :} \large}

\newcommand{\tm}{\large \textbf{Théoreme :} \large}

\newcommand{\ex}{\textcolor{green}{Exemple :} }

\newcommand{\dm}{\textcolor{red}{\textbf{Démo :} } }

\newcommand{\de}{\large \textbf{Définition} \large }

\newcommand{\rmq}{\textbf{Remarque :} }

\newcommand{\bs}{\bigskip}

\newcommand{\voca}{\textcolor{blue}{\textbf{Vocabulaire} } }

\newcommand{\lem}{\textcolor{red}{\textbf{Lemme :} } }

\newcommand{\trinom}[3]{\begin{pmatrix}
    #1 \\
    #2 \\
    #3
\end{pmatrix}}

\newcommand{\quadrinom}[4]{\begin{pmatrix}
    #1 \\
    #2 \\
    #3 \\
    #4 \\
\end{pmatrix}}

\newcommand{\pentanom}[5]{\begin{pmatrix}
    #1 \\
    #2 \\
    #3 \\
    #4 \\
    #5
\end{pmatrix}}

\newcommand{\hexanom}[6]{\begin{pmatrix}
    #1 \\
    #2 \\
    #3 \\
    #4 \\
    #5 \\
    #6 
\end{pmatrix}}

\newcommand{\serie}[2]{\displaystyle\sum_{#1 =0}^{+\infty} #2_{#1} }

\newcommand{\tend}{\underset{n \to + \infty}{\longrightarrow} }

\newcommand{\Lra}{\Leftrightarrow}

\newcommand{\lra}{\leftrightarrow}

\newcommand{\Ra}{\Rightarrow}

\newcommand{\ra}{\rightarrow}

\newcommand{\la}{\leftarrow}

\newcommand{\La}{\Leftarrow}

\newcommand{\dsum}[2]{\displaystyle\sum_{#1}^{#2} }

\newcommand{\dint}[2]{\displaystyle\int_{#1}^{#2} }

\newcommand{\ntend}{\underset{n \to + \infty}{\not \longrightarrow} }

\newenvironment{lmatrix}{$ \left|\begin{array}{l} }{\end{array}\right.$}

\newcommand{\img}[4]{\begin{figure}[!ht]
    \centering
    \includegraphics[scale=#1 ]{#2}
    \caption{#3}
    \label{#4}
    \end{figure} }    
\begin{document}

\newcommand{\grad}[1]{\vv{grad}#1}


\title{Option info}
\author{Schobert Néo}

\maketitle

\tableofcontents

\newpage 


\section{Les bases}



\begin{itemize}
    \item Comment afficher une chaine de caractères.
    \item Comment afficher un saut de ligne.
    \item Comment concaténer deux chaines. Exemple : concaténer "liberté, " "égalité, " "fraternité, ".
    \item Que fait la fonction nth\_char.
    \item Que fait la fonction sub\_string.
    \item Qu'est-ce qu'une fonction curryfiée ?
    \item Quelles sont les opérations de base sur les couples ?
    \item Quel ordre utilise la comparaison des chaines de caractères ? 
    \item Qu'est-ce qu'un type produit ? 
    \item Qu'est-ce qu'un nom avec paramètre ? 
    \item Définir le type récursif. L'utiliser dans le cas des couleurs.
    \item Ecrire une fonction qui calcule la composante RGB d'une couleur.
    \item Différence entre function et fun.
    \item Que ne faut-il pas faire dans le pattern matching ? (ordre comptes)
    \item Que fait l'opération prefix ?
    \item Définir la fonction égal en utilisant prefix.
    \item Comment lever une erreur ?
    \item Comment définir deux fonctions mutuellement récursives ?
    \item Refaire l'exo 01.3.
\end{itemize}



\section{Les listes}


\begin{itemize}
    \item Reconstruire le type liste.
    \item Comment obtenir la tête et la queue d'une liste ?
    \item Comment changer un opérateur infix en opérateur préfix ?
    \item 
\end{itemize}




\end{document}