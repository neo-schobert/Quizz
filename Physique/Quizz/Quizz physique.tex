\documentclass[a4paper, 11pt, hidelinks]{article}
\usepackage{bookmark}
\usepackage[utf8]{inputenc} 
\usepackage[T1]{fontenc}
\usepackage{lmodern}
\usepackage{graphicx}
\usepackage[french]{babel}
\usepackage{geometry}
\usepackage{eucal}
\usepackage{caption}
\usepackage{float}
\usepackage{url}
\usepackage{amsmath}
\usepackage{amssymb}
\usepackage{color}
\usepackage{hyperref}
\usepackage{cancel}
\usepackage{tikz}
\usepackage{mathrsfs}  
\usepackage{esvect}


\geometry{hmargin=2cm,vmargin=1.5cm}

\tikzset{
  treenode/.style = {shape=rectangle, rounded corners,
                     draw, align=center,
                     top color=white, bottom color=blue!5},
  root/.style     = {treenode, font=\Large, bottom color=red!10},
  env/.style      = {treenode, font=\ttfamily\normalsize},
  dummy/.style    = {circle,draw}
}

\newcommand{\prp}{\large \textbf{Proposition :} \large}

\newcommand{\tm}{\large \textbf{Théoreme :} \large}

\newcommand{\ex}{\textcolor{green}{Exemple :} }

\newcommand{\dm}{\textcolor{red}{\textbf{Démo :} } }

\newcommand{\de}{\large \textbf{Définition} \large }

\newcommand{\rmq}{\textbf{Remarque :} }

\newcommand{\bs}{\bigskip}

\newcommand{\voca}{\textcolor{blue}{\textbf{Vocabulaire} } }

\newcommand{\lem}{\textcolor{red}{\textbf{Lemme :} } }

\newcommand{\trinom}[3]{\begin{pmatrix}
    #1 \\
    #2 \\
    #3
\end{pmatrix}}

\newcommand{\quadrinom}[4]{\begin{pmatrix}
    #1 \\
    #2 \\
    #3 \\
    #4 \\
\end{pmatrix}}

\newcommand{\pentanom}[5]{\begin{pmatrix}
    #1 \\
    #2 \\
    #3 \\
    #4 \\
    #5
\end{pmatrix}}

\newcommand{\hexanom}[6]{\begin{pmatrix}
    #1 \\
    #2 \\
    #3 \\
    #4 \\
    #5 \\
    #6 
\end{pmatrix}}

\newcommand{\serie}[2]{\displaystyle\sum_{#1 =0}^{+\infty} #2_{#1} }

\newcommand{\tend}{\underset{n \to + \infty}{\longrightarrow} }

\newcommand{\Lra}{\Leftrightarrow}

\newcommand{\lra}{\leftrightarrow}

\newcommand{\Ra}{\Rightarrow}

\newcommand{\ra}{\rightarrow}

\newcommand{\la}{\leftarrow}

\newcommand{\La}{\Leftarrow}

\newcommand{\dsum}[2]{\displaystyle\sum_{#1}^{#2} }

\newcommand{\dint}[2]{\displaystyle\int_{#1}^{#2} }

\newcommand{\ntend}{\underset{n \to + \infty}{\not \longrightarrow} }

\newenvironment{lmatrix}{$ \left|\begin{array}{l} }{\end{array}\right.$}

\newcommand{\img}[4]{\begin{figure}[!ht]
    \centering
    \includegraphics[scale=#1 ]{#2}
    \caption{#3}
    \label{#4}
    \end{figure} }    
\begin{document}

\newcommand{\grad}[1]{\vv{grad}#1}


\title{Physique 18-10}
\author{Schobert Néo}

\maketitle

\tableofcontents

\newpage 


\section{Ensemble des chapitres :}
\cite{Chapitre1}
\cite{Chapitre2}
\cite{Chapitre3}
\cite{Chapitre4}
\cite{Chapitre5}
\cite{Chapitre6}
\cite{Chapitre7}
\cite{Chapitre8}
\cite{Chapitre9}
\cite{Chapitre10}
\cite{Chapitre11}
\cite{Chapitre12}
\cite{Chapitre13}
\cite{Chapitre14}
\cite{Chapitre15}
\cite{Chapitre16}
\cite{Chapitre17}
\cite{Chapitre18}
\cite{Chapitre19}
\cite{Chapitre20}
\cite{Chapitre21}


\section{18 octobre}




\subsection{Question :}

\begin{itemize}
    \item Qu'est-ce que le blanc d'ordre supérieur ? \cite{Chapitre8}
    \item Qu'est-ce que la frange achromatique ? \cite{Chapitre8}
    \item Qu'est-ce que les teintes de Newton ? \cite{Chapitre8}
    \item Qu'est qu'un objet de phase ? \cite{Chapitre8}
    \item Qu'est-ce qu'un réseau ? \cite{Chapitre6}
    \item Qu'est-ce que le pas du réseau \cite{Chapitre6}
    \item Relation nombre de trait par unité de longueur / pas \cite{Chapitre6}
    \item Comment sont caractérisés chaque trait d'un réseau ? et vers où emettent ces traits ? \cite{Chapitre6}
    \item Qu'observe-t-on après l'éclairage d'un réseau par un laser Hélium-Néon ? \cite{Chapitre6}
    \item \img{0.5}{Images/Reseau en transmission.PNG}{Réseau en transmission}{Figure 1}
    Différence de marche entre deux ondes consécutives. \cite{Chapitre6}
    \item Rappeler la relation fondamentale des réseaux en transmission. \cite{Chapitre6}
    \item Qu'est-ce que l'ordre d'interférence du réseau ? \cite{Chapitre6}
    \item Pour quel ordre obtient-on les conditions de l'optique géométrique ? \cite{Chapitre6}
    \newpage
    \item \img{0.5}{Images/Reseau en reflexion.PNG}{Réseau en réflexion}{Figure 2}
    Rappeler la relation fondamentale des réseaux en reflexion. \cite{Chapitre6}
    \item Quelle expérience met en évidence la présence d'un minimum de déviation ? \cite{Chapitre6}
    \item Comment s'écrit la déviation ? Comment l'exploiter. \cite{Chapitre6}
    \item Cas intéressant de $\theta_K$ en fonction de $\theta_0$. \cite{Chapitre6}
    \item Relation du minimum de déviation. \cite{Chapitre6}
    \item \textcolor{red}{Reprendre ici}
    \item Qu'est-ce que le pouvoir dispersif d'un réseau ? Donner la formule \cite{Chapitre6}
    \item Pour un réseau éclairé en incidence normale $\theta_0=0$ par deux radiations $\lambda_1<\lambda_2$,
    que dire; selon $K$, de la déviation ? \cite{Chapitre6}
    \item A quel cas peut-on opposer cela ? (Cas du prisme) \cite{Chapitre6}
    \item Valeur de $\Delta \varphi$ \cite{Chapitre6}
    \item Méthode pour calculer la fonction de réseau. \cite{Chapitre6} 
    \item Formule fonction de réseau et intensité. \cite{Chapitre6}
    \item Périodicité de $R(\Delta \varphi)$ \cite{Chapitre6}
    \item Cas d'annulation de $R(\Delta \varphi)$ \cite{Chapitre6}
    \item Cas des maximas primaires : Conditions et conséquences pour $R(\Delta \varphi)$. \cite{Chapitre6}
    \item Que réprésente $N$ ? \cite{Chapitre6}
    \item $R(\Delta \varphi)$ dans le cas des maximas primaires et largeur du pic central. \cite{Chapitre6}
    \item Valeur de $\Delta (\Delta \varphi)$ largeur du pic central. \cite{Chapitre6}
\end{itemize}



\subsection{Remarques}
\begin{itemize}
    \item Pourquoi $\theta_0$ dépend de $\theta_K$ ? \cite{Chapitre6}
    \item Pourquoi $R(\Delta \varphi)$ est $2\pi$-périodique et pas $\frac{2\pi}{N}$-périodique ? \cite{Chapitre6} 
\end{itemize}



\newpage

\section{19 octobre}


\subsection{Question :}

\begin{itemize}
    \item Par rapport à quelle variable peut-on écrire R la fonction de réseau ? \cite{Chapitre6}
    \item Valeur de la largeur du pic central par rapport à $\Delta (sin(\theta))$ \cite{Chapitre6}
    \img{0.5}{Images/Resolution totale de deux radiations lambda1 et lambda2.PNG}{Résolution totale de deux radiations $\lambda_1$ et $\lambda_2$}{Figure 4}
    \item Cas du pouvoir séparateur d'un réseau. Valeurs des pics des doublets de sodium, largeur du pic et distance entre les deux pics. \cite{Chapitre6}
    \item Qu'est-ce que le critère de Rayleigh ? \cite{Chapitre6}
    \item Que vaut les paramètres dans le cas limite d'après le critère de Rayleigh ? \cite{Chapitre6}
    \item Définir le pouvoir de résolution. \cite{Chapitre6}
    \item Expression de la loi de Coulomb \cite{Chapitre9}
    \img{0.5}{Images/Definition de la loi de Coulomb (cas q1q2 superieur a 0).PNG}{Schéma de la loi de Coulomb (cas $q_1q_2>0$)}{Figure 5}
    \item Valeur de la permittivité dans un milieu autre que le vide. \cite{Chapitre9} diélectrique du vide \cite{Chapitre9}
    \item Analogie Loi de Coulomb / Gravitation \cite{Chapitre9}
    \item Qu'est-ce que le rayon de Bohr ? \cite{Chapitre9}
    \item Valeur du rayon de Bohr \cite{Chapitre9}
    \item Lien champ électrostatique / force électrostatique.(charge ponctuelle q) \cite{Chapitre9}
    \item Lien champ électrostatique / force électrostatique.($n$ charges ponctuelles $q_i$) \cite{Chapitre9}
    \item Exemple du champ produit par un triangle équilatéral dont les sommets sont de charges $(2q,q,q)$ \cite{Chapitre9}
    \item Que vaut la charge portée à la surface d'une sphère métallique en cuivre de rayon R portée à un potentiel $V_1$. \cite{Chapitre9}
    \item Comment calculer le nombre de charges négatives en défaut et le nombre total de charges mobiles ? \cite{Chapitre9} 
    \item Comment caractériser la charge dans la matière électrisée. Quel lien avec la continuité ? \cite{Chapitre9}
    \item Définir densité linéïque de charge / densité surfacique de charge / densité volumique de charge. \cite{Chapitre9}
    \item Lien entre les trois modèles (cas du fil rectiligne de rayon $r$ chargé) \cite{Chapitre9}
    \item Expression élémentaire des champs (en fonction des densités linéiques / surfaciques / volumiques de charge) \cite{Chapitre9}
    \newpage
    \img{0.5}{Images/Plan de symetrie d'une distribution de charge.PNG}{Plan de symétrie d'une distribution de charge}{Figure 6}
    \item Qu'est-ce qu'un plan de symétrie pour une distribution de charge. (Et sa notation) \cite{Chapitre9}
    \item Comment calculer $d\vv{E}(M)$ ? \cite{Chapitre9}
    \item Conséquences d'un plan de symétrie sur le champ en deux points symétriques $M$ et $M'$. \cite{Chapitre9}
    \img{0.5}{Images/Plan d'antisymetrie d'une distribution de charge.PNG}{Plan d'antisymétrie d'une distribution de charge}{Figure 7}
    \item Qu'est-ce qu'un plan d'antisymétrie pour une distribution de charge. (Et sa notation) \cite{Chapitre9}
    \item Comment calculer $d\vv{E}(M)$ ? \cite{Chapitre9}
    \item Conséquences d'un plan d'antisymétrie sur le champ en deux points symétriques $M$ et $M'$. \cite{Chapitre9}
    \item Qu'est-ce qu'une transformation isométrique du champ ? \cite{Chapitre9}
    \item Comment se transforme le champ lors d'une transformation isométrique ? \cite{Chapitre9}
    \item Que peut-on dire de l'invariance par translation ? \cite{Chapitre9}
    \item Que peut-on dire du champ lors d'une translation selon un axe ? \cite{Chapitre9}
    \item Que signifie l'invariance par rotation ? \cite{Chapitre9}
    \item Que peut-on dire du champ lors d'une rotation selon un axe ? \cite{Chapitre9}
    \item Pourquoi il ne faut pas établir un lien entre la dépendance du champ et sa direction ? \cite{Chapitre9}
    \item Rappeler le principe général de Curie. \cite{Chapitre9}
    \item Qu'est-ce que la circulation du champ électrique ? \cite{Chapitre9}
    \item Donner la formule de la circulation du champ $\vv{E}$ entre le point $A$ et $B$. \cite{Chapitre9}
    \item Que dire de la circulation du champ ? \cite{Chapitre9}
    \item Qu'est-ce que le potentiel électrostatique ? \cite{Chapitre9}
    \item Valeur du potentiel électrostatique (charge ponctuelle) \cite{Chapitre9}
    \item Lien potentiel électrostatique / circulation du champ \cite{Chapitre9}
    \item Valeur du potentiel électrostatique (distribution de $n$ charges $q_i$) \cite{Chapitre9}
    \item Lien circulation du champ / Différence de potentiel entre deux points. \cite{Chapitre9}
\end{itemize}






\subsection{Remarques}
\begin{itemize}
    \item Peut-on avoir une distribution de charge non homogène.
    \item Existe-il une sorte d'aimant à charge ?
    \item Si oui, on peut imaginer un recouvrement systématique par translation vraie en pratique.
\end{itemize}



\section{8 Novembre}

\subsection{Questions}

\begin{itemize}
    \item Superposition du potentiel électrostatique \cite{Chapitre9}
    \item Lien entre champ et potentiel. \cite{Chapitre9}
    \item Définir le gradient. \cite{Chapitre9}
    \item Lien entre $dV$ et $\vv{grad} V$ \cite{Chapitre9}
    \item Définir le signe Nabla $\nabla$ \cite{Chapitre9}
    \item Définir gradient en coordonnées cartésiennes / cylindriques / sphériques. \cite{Chapitre9}
    \item Sens physique du gradient. \cite{Gradient} 
    \item Travail en fonction du potentiel électrostatique. \cite{Chapitre9}
    \item Définir l'énergie potentielle électrostatique \cite{Chapitre9}
    \item Cas d'une distribution de charges \cite{Chapitre9}
    \item Notion de surface orientée. \cite{Chapitre9}
    \item Définir flux élémentaire en surface ouverte. \cite{Chapitre9}
    \item Définir flux en surface fermée. \cite{Chapitre9}
    \item Cas du flux élémentaires créé par une charge ponctuelle. (surface ouverte)\cite{Chapitre9}
    \item Définir l'angle solide élémentaire. \cite{Chapitre9}
    \item Valeur de l'angle solide total. \cite{Chapitre9}
    \item Cas du flux élémentaire créé par une charge ponctuelle. (surface fermée) \cite{Chapitre9}
    \item Enoncer le théorème de Gauss. \cite{Chapitre9}
    \item Rappeler les stratégies de mise en oeuvre. \cite{Chapitre9}
    \item Rappeler toutes les putains de conditions sur le théorème de Gauss. \cite{Chapitre9}
    \item Lien entre norme du champ par rapport à $Q_{int}$ et $S_i$. \cite{Chapitre9}
    \item Comment choisir S la surface ? \cite{Chapitre9}
    \item Se remémorer les exos d'applications (a,b,c) (sphère / cylindre / \textbf{plan $1\varphi$ni chargé en $z=0$}) \cite{Chapitre9}
\end{itemize}






\subsection{Remarques}

\begin{itemize}
    \item $f(x)=-\frac{dE_p(x)}{dx}\vv{u}_x$ dans le cas de forces conservatives. Condition nécessaire
    d'ailleurs pour qu'une force soit conservative. Cette relation peut-elle être assimilée dans le cas 
    de plusieurs variables à : $f(x,y,z)=-\vv{grad}E_p$
\end{itemize}



\section{9 Novembre}

\subsection{Questions}

\begin{itemize}
    \item On a continuité de la composante tangentielle et discontinuité de la composante normale. \cite{Chapitre9}
    \item Comment obtenir le potentiel à partir du champ ? \cite{Chapitre9}
    \item Pourqoi le potentiel doit être continue ? \cite{Chapitre9}
    \item Intérêt de la continuité du potentiel. \cite{Chapitre9}
    \item Se remémorer les trois exemples. \cite{Chapitre9}
    \item Définir un condensateur. \cite{Chapitre9}
    \item Définir une armature. \cite{Chapitre9}
    \item A quoi ressemble un condensateur en influence totale. \cite{Chapitre9}
    \item Les charges doivent être au repos; qu'est-ce que ça implique sur le champ. \cite{Chapitre9}
    \item Utiliser Gauss pour avoir la répartition des charges dans un condensateur. \cite{Chapitre9}
    \item Définir la capacité d'un condensateur. \cite{Chapitre9}
    \item Définir en pratique un condensateur plan. \cite{Chapitre9}
    \item Conditions pratiques d'un condensateur plan. \cite{Chapitre9}
    \item Calculer le champ produit par un condensateur plan. \cite{Chapitre9}
    \item Determiner la capacité d'un condensateur plan. \cite{Chapitre9}
    \item Valeur de la capacité d'un condensateur plan. \cite{Chapitre9}
    \item Valeur de la permittivité dans un milieu autre que le vide. \cite{Chapitre9}
    \item Lois d'association des condensateurs / mnémotechnique. \cite{Chapitre9}
    \item Calculer l'énergie électrique emmagasinée dans un condensateur. \cite{Chapitre9}
    \item Comment obtenir les équations des lignes de champ. \cite{Chapitre9}
    \item Deux possibilités lorsque deux lignes de champ se coupent. Citer un exemple pour chacun. \cite{Chapitre9}
    \item Définir un tube de champ. \cite{Chapitre9}
    \item Qu'est-ce qu'une zone isopotentielle ? \cite{Chapitre9}
    \item Propriété du champ par rapport aux zones isopotentielles. \cite{Chapitre9}
    \item Comment est orienté le champ électrostatique ? \cite{Chapitre9}
    \item Qu'est-ce que le resserrement ou l'évasement ? \cite{Chapitre9}
    \item Que ce passe-t-il pour l'intensité du champ électrostatique lors d'un évasement / resserement ? \cite{Chapitre9}
    \item Exemple de cartes de champs. Trouver les symétries et les zones équipotentielles. \cite{Chapitre9}
    \item Calculer le flux élémentaire du champ électrique à travers la surface fermée du méso-cube. \cite{Chapitre10}
    \item Définir la divergence. ($div(\vv{E}))$ \cite{Chapitre10}
    \item Qu'est ce que l'équation de Maxwell-Gauss. \cite{Chapitre10}
    \item Citer le théorème de Green-Ostrogradski. \cite{Chapitre10} 
    \item Utilité de ce théorème. \cite{Chapitre10}
    \item Comment passer de la forme intégrale à la forme locale du théorème de Gauss ? \cite{Chapitre10}
    \item Donner la signification de la divergence. \cite{Chapitre10}
    \item Calculer la circulation élémentaire du champ électrostatique sur le contour fermé. \cite{Chapitre10}
    \item Définir la rotationnelle. ($\vv{rot}\vv{E})$ \cite{Chapitre10}
    \item Equation de Maxwell-Faraday de la statique. \cite{Chapitre10}
    \item Autre expression de $\vv{rot}\vv{E}$. \cite{Chapitre10}
    \item Que remarque-t-on pour mémoriser plus facilement l'expression de $\vv{rot}\vv{E}$. \cite{Chapitre10}
    \item Citer le théorème de Stokes-Ampère. \cite{Chapitre10}
    \item Utilité de ce théorème. \cite{Chapitre10}
    \item Donner la signification de la rotationnelle. \cite{Chapitre10}
\end{itemize}






\subsection{Remarques}

\begin{itemize}
    \item $dV=\grad{V}.\vv{dr}$. S'agit-il simplement d'une dérivée directionnelle :
    
    $dV=\lim\limits_{h \rightarrow 0} \frac{V(\vv{M}+h\vv{dr})-V(\vv{M})}{h}$
    \item Pas compris la notation suivante :
    
    $V(B)-V(A)=dV_{\vv{dr}}=\grad{V}.\vv{dr}$
    \item La divergence au final, c'est la dérivée directionnelle donnée par le vecteur $(1,1,1)$
\end{itemize}


\section{15 Novembre}



\subsection{Questions}

\begin{itemize}
    \item Qu'est-ce que le Laplacien ? Définition avec les dérivées et avec le div. \cite{Chapitre10}
    \item Définir le Laplacien vectoriel \cite{Chapitre10}
    \item Equation de Poisson. Expression et preuve. \cite{Chapitre10}
    \item Résumé du problème de Dériclé. \cite{Chapitre10}
    \item Que peut-on dire quand une dimension est très grande devant une autre ? \cite{Chapitre10}
    \item Analogie gravitation / Electrostatique. \cite{Chapitre10}
    \item Définition du vecteur densité volumique de courant. \cite{Chapitre11}
    \item Donner l'expression du courant. \cite{Chapitre11}
    \item Valeur du vecteur densité de courant dans le cas de plusieurs types de porteurs de charge. \cite{Chapitre11}
    \item Définition véritable du vecteur densité de courant. \cite{Chapitre11}
    \item $\vv{J}(M)$ en fonction de $I(M)$ \cite{Chapitre11}
    \item $\vv{J}(M)$ en fonction de $\vv{v}(M)$ \cite{Chapitre11}
    \item Equivalence $1D$-$3D$. Ecrire $\vv{J}(M) d \tau$ \cite{Chapitre11}
    \item Cas de la distribution surfacique. Et définition du vecteur densité surfacique de courant. \cite{Chapitre11}
    \item Rappeler le principe de Curie. \cite{Chapitre11}
    \item Donner la loi de Biot et Savart. $d\vv{B}_P(M)=\frac{\mu_0}{4 \pi}\frac{\vv{J}(P)\land \vv{PM}}{PM^3}d\tau$ \cite{Chapitre11}
\end{itemize}




\subsection{Remarques}


\begin{itemize}
    \item 
\end{itemize}



\section{16 Novembre}


\subsection{Questions}

\begin{itemize}
    \item Quelle est la conséquence ? \cite{Chapitre11}
    \item Dans un plan de symétrie $\Pi^+$, comment se comporte le vecteur force, le vecteur vitesse, et le vecteur champ magnétique. \cite{Chapitre11}
    \item Faire un léger rapprochement (largement faux) entre le champ magnétique et le champ électrostatique. L'un dans le cas d'un $\Pi^+$, l'autre dans le cas d'un $\Pi^-$. \cite{Chapitre11}
    \item Dans un plan d'antisymétrie $\Pi^-$, comment se comporte le vecteur force, le vecteur vitesse, et le vecteur champ magnétique. \cite{Chapitre11}
    \item Faire un léger rapprochement (largement faux) entre le champ magnétique et le champ électrostatique. L'un dans le cas d'un $\Pi^-$, l'autre dans le cas d'un $\Pi^+$. \cite{Chapitre11}
    \item Quelle rapprochement peut-on faire entre les invariances dans le champ magnétiques et celles dans le champ électrostatique. \cite{Chapitre11}
    \item Sur quoi exactement se base le principe de Curie dans ce cours ? \cite{Chapitre11}
    \item Que ce passe-il pour le champs magnétique lors d'une translation; d'une rotation ? \cite{Chapitre11}
    \item Définir le flux magnétostatique \cite{Chapitre11}
    \item Que peut-on dire du flux magnétostatique. \cite{Chapitre11}
    \item Quel est le lien avec la divergence de $\vv{B}$ \cite{Chapitre11}
    \item Equation de Maxwell-Thomson. \cite{Chapitre11}
    \item Valeur de $\vv{B}$ grâce à la loi de Biot Savart. \cite{Chapitre11}
    \item Définition de la circulation du champ magnétique. \cite{Chapitre11}
    \item Discussion en fonction de $\Gamma$ \cite{Chapitre11}
    \item Citer le théorème d'Ampère. \cite{Chapitre11}
    \item Valeur de la perméabilité magnétique du vide. \cite{Chapitre11}
    \item Que vaut $I_{enlace}$ dans le cas d'une distribution filiforme / volumique / surfacique. \cite{Chapitre11}
    \item Donner la stratégie de mise en \oe uvre. \cite{Chapitre11}
    \item Rappeler les conditions pour appliquer le théorème d'Ampère "idéal". \cite{Chapitre11}
    \item Que peut-on dire du champ magnétique ? \cite{Chapitre11}
    \item Comment faire pour utiliser Ampère dans le cas du solénoide infini / de la nappe de courant ? \cite{Chapitre11}
    \item Rappeler l'équation de Maxwell-Ampère et sa "preuve". \cite{Chapitre11}
    \item Autour de quoi tourne le courant magnétostatique ? \cite{Chapitre11}
    \item Que se passe-t-il pour le champ magnétostatique lors d'un évasement / resserrement. \cite{Chapitre11}
\end{itemize}



\subsection{Remarques}


\begin{itemize}
    \item Si un fil est enlacé 2 fois ? (c'est bon en fait)
    \item Continuité du champ magnétique completement pété dans le cas du solénoide infini.
    \item 
\end{itemize}




\section{19 Novembre}

\subsection{Questions}



\begin{itemize}
    \item Définir un dipole \cite{Chapitre12}
    \item Définir le moment dipolaire \cite{Chapitre12}
    \item Moment dipolaire dans le cas de $n$ charges. (Voir chapitre 12 Fiches)
    \item Définir le Debye. \cite{Chapitre12}
    \item Définition du barycentre. \cite{Chapitre12}
    \item Calcul du potentiel électrostatique en approximation dipolaire. \cite{Chapitre12}
    \item Valeur du potentiel électrostatique en approximation dipolaire. \cite{Chapitre12}
    \item Valeur du champ électrostatique dipolaire. \cite{Chapitre12}
    \item Calculer le champ électrostatique dipolaire. \cite{Chapitre12}
    \item Définir les positions de Gauss. \cite{Chapitre12}
    \item Trouver l'équation des lignes de champs. \cite{Chapitre12}
    \item Trouver l'équation des isopotentielles. \cite{Chapitre12}
    \item Calculer le moment et la résultante des actions subies par un dipole plongé dans un champ électrostatique uniforme. \cite{Chapitre12}
    \item Calculer le moment et la résultante des actions subies par un dipole plongé dans un champ électrostatique non uniforme. \cite{Chapitre12}
\end{itemize}



\subsection{Remarques}


\begin{itemize}
    \item Existe-t-il une formule reliant $\chi$ et $p$ ?
\end{itemize}



\section{22 Novembre}



\subsection{Questions}


\begin{itemize}
    \item Lien entre la force et l'énergie potentielle. \cite{Chapitre12}
    \item Inexistence du monopole magnétique. \cite{Chapitre12}
    \item On a que des dipoles. \cite{Chapitre12}
    \item Comment le montrer ? \cite{Chapitre12}
    \item Définition du moment magnétique. \cite{Chapitre12}
    \item unité du moment magnétique. \cite{Chapitre12}
    \item Moment cinétique électronique. \cite{Chapitre12}
    \item Moment dipolaire électronique. \cite{Chapitre12}
    \item Rapport gyromagnétique de l'électron. \cite{Chapitre12}
    \item Idée de moment de spin. \cite{Chapitre12}
    \item Définition du magnéton de Bohr. \cite{Chapitre12}
    \item Ordre de grandeurs de moments magnétiques. \cite{Chapitre12}
    \item Analogie entre électrique et magnétique. \cite{Chapitre12}
    \item Retrouver l'équation des lignes de champs. \cite{Chapitre12}
    \item Retrouver la valeur des actions mécaniques subiées par un dipôle magnétique plongé dans un champ magnétique extérieur uniforme. \cite{Chapitre12}
    \item Valeur des actions mécaniques subiées par un dipôle magnétique plongé dans un champ magnétique extérieur non uniforme \cite{Chapitre12}
    \item Notion du flux coupé. \cite{Chapitre12}
    \item Travail des forces de Laplace sur un circuit lors du déplacement $\vv{dr}$ \cite{Chapitre12}
    \item Théorème de Maxwell \cite{Chapitre12}
    \item Règle du flux maximal. \cite{Chapitre12}
    \item Equation de la conservation de la charge en $1D$. \cite{Chapitre13}
    \item Equation de la conservation de la charge en $3D$. \cite{Chapitre13}
\end{itemize}




\subsection{Remarques}


\begin{itemize}
    \item Dans le cadre non statique, le monopole magnétique peut-il exister ?
    \item J'ai rien capté à l'histoire des $m_\ell$
\end{itemize}



\section{23 Novembre}


\subsection{Questions}


\begin{itemize}
    \item Retrouver la loi des noeuds en ARQS. \cite{Chapitre13}
    \item Retrouver l'équation de Maxwell-Ampère. \cite{Chapitre13}
    \item Que peut-on dire de l'intensité dans le condensateur. \cite{Chapitre13}
    \item Visualiser les effets du champ électrostatique sur le champ magnétique. (transport d'électricité) \cite{Chapitre13}
    \item Rappeler le phénomène d'induction. \cite{Chapitre13}
    \item Différence entre induction de Newmann et induction de Lorentz. \cite{Chapitre13}
    \item Retrouver la force électromotrice. \cite{Chapitre13}
    \item Rappeler la loi de Lenz-Faraday. \cite{Chapitre13}
    \item Retrouver l'équation de Maxwell-Faraday. \cite{Chapitre13}
    \item Donner les 4 équations de Maxwell en local et en global. \cite{Chapitre13}
    \item Valeur de la perméabilité du vide. \cite{Chapitre13}
    \item Valeur de la permittivité diélectrique du vide. \cite{Chapitre13}
    \item Lien entre $\mu_0$ et $\epsilon_0$ \cite{Chapitre13}
    \item Définition de l'ARQS. Ses critères de validité à redémontrer. \cite{Chapitre13}
    \item Bilan des équation de Maxwell en ARQS magnétique. \cite{Chapitre13}
    \item Définition de l'ARQS électrique. Ses caractères de validité à redémontrer. \cite{Chapitre13}
    \item Bilan des équations de Maxwell en ARQS électrique. \cite{Chapitre13}
\end{itemize}




\subsection{Remarques}


\begin{itemize}
    \item Pour démontrer Maxwell-Faraday, on a utilisé le flux. Mais il est nul d'après Maxwell-Thompson.
    \item Qu'est-ce qui nous prouve que les champs engendrés convergent.
    \item $B_1<B_0$ quoi qu'il arrive.
\end{itemize}




\section{29 Novembre}



\subsection{Questions}

\begin{itemize}
    \item Quelles équations permettent de déduire que $\vv{E}$ et $\vv{B}$ sont couplés. \cite{Chapitre13}
    \item Quelles équations sont constitutives des champs $\vv{E}$ et $\vv{B}$ \cite{Chapitre13}
    \item Retrouver l'équation de d'Alembert pour le champ $\vv{E}$ et pour le champ $\vv{B}$ \cite{Chapitre13}
    \item Retrouver la loi D'Ohm locale. \cite{Chapitre14}
    \item Ordre de grandeurs de $\gamma$ \cite{Chapitre14}
    \item Retrouver la valeur de la résistance cas général et dans le cas d'un conducteur ohmique cylindrique de sections $S$ droite et de longueur $L$. \cite{Chapitre14}
    \item Retrouver la puissance cédée aux porteurs de charge. \cite{Chapitre14}
    \item Retrouver les 2 causes de variation de l'energie du champ électromagnétique. \cite{Chapitre14}
    \item Retrouver l'identité de Poynting. \cite{Chapitre14}
    \item Valeur du vecteur de Poynting. \cite{Chapitre14}
    \item Qu'est-ce que la densité volumique d'énergie électromagnétique / électrique / magnétique. \cite{Chapitre14}
    \item Théorème de Poynting. \cite{Chapitre14}
    \item Ordre de grandeur de flux surfaciques. \cite{Chapitre14}
    \item Ordre de grandeur $\frac{\epsilon_m}{\epsilon_e}$ \cite{Chapitre14}
\end{itemize}




\subsection{Remarques}

\begin{itemize}
    \item La puissance rayonnée est relative au volume ou au champ électromagnétique ?
    \item 
\end{itemize}



\section{30 Novembre}

\subsection{Questions}

\begin{itemize}
    \item Retrouver l'EDA 1D (corde) \cite{Chapitre15}
    \item Retrouver l'EDA 1D (Câble coaxial) \cite{Chapitre15}
    \item Quelles sont les variables "bonnes sa mère". Et pourquoi elles sont trop bonnes. \cite{Chapitre15}
    \item Retrouver l'EDA 1D avec les bonnes variables. \cite{Chapitre15}
    \item 
\end{itemize}




\subsection{Remarques}


\begin{itemize}
    \item Notation complexe dérivée partielle.
\end{itemize}





\section{2 Décembre}


\subsection{Questions}


\begin{itemize}
    \item 
\end{itemize}


\subsection{Remarques}


\begin{itemize}
    \item 
\end{itemize}


\section{6 Décembre}



\subsection{Questions}



\begin{itemize}
    \item Définir une onde plane. \cite{Chapitre15}
    \item Qu'est-ce que le plan d'onde. \cite{Chapitre15}
    \item Comment passer de l'EDA 3D à l'EDA 1D ? \cite{Chapitre15}
    \item Définition de l'OPPH. \cite{Chapitre15}
    \item Problème de l'OPPH. \cite{Chapitre15}
    \item Utilité de la synthèse de Fourrier sur l'OPPH. \cite{Chapitre15}
    \item Ecrire la synthèse de Fourrier sur l'OPPH. \cite{Chapitre15}
    \item Ordre de grandeur Spectre électromagnétique. \cite{Chapitre15}
    \item Définit la vitesse de phase. \cite{Chapitre15}
    \item Qu'est-ce qu'un milieu non dispersif. \cite{Chapitre15}
    \item Transformation des opérateurs. \cite{Chapitre15}
    \item Equation de Maxwell avec les opérateurs en complexe. \cite{Chapitre15}
    \item Qu'est-ce que la relation de structure de l'onde plane. \cite{Chapitre15}
    \item Qu'est-ce que l'étude de la polarisation d'une OPPH. \cite{Chapitre15}
    \item Mener l'étude sur le champ électrique. \cite{Chapitre15}
    \item Définir polarisation rectiligne et circulaire. \cite{Chapitre15}
    \item Faire l'énergétique d'une OPPH. \cite{Chapitre15}
    \item Retrouver la vitesse de transport de l'énergie d'un OEM. \cite{Chapitre15}
    \item Comment calculer l'énergie d'un un volume élémentaire. ($2$ façons.) \cite{Chapitre15}
    \item Valeur moyenne en complexe. \cite{Chapitre15}
    \item Définir un plasma \cite{Chapitre16}
    \item 
\end{itemize}


\subsection{Remarques}



\begin{itemize}
    \item Pour déterminer le sens de parcourt de l'onde dans le cas $\varphi=\frac{\pi}{2}$, pouvait-on utiliser le rotationnel.
\end{itemize}







% \subsection{Questions}



% \begin{itemize}
%     \item 
% \end{itemize}


% \subsection{Remarques}



% \begin{itemize}
%     \item 
% \end{itemize}






% \subsection{Questions}



% \begin{itemize}
%     \item 
% \end{itemize}


% \subsection{Remarques}



% \begin{itemize}
%     \item 
% \end{itemize}









% \subsection{Questions}



% \begin{itemize}
%     \item 
% \end{itemize}


% \subsection{Remarques}



% \begin{itemize}
%     \item 
% \end{itemize}









% \subsection{Questions}



% \begin{itemize}
%     \item 
% \end{itemize}


% \subsection{Remarques}



% \begin{itemize}
%     \item 
% \end{itemize}








% \subsection{Questions}



% \begin{itemize}
%     \item 
% \end{itemize}


% \subsection{Remarques}



% \begin{itemize}
%     \item 
% \end{itemize}









% \subsection{Questions}



% \begin{itemize}
%     \item 
% \end{itemize}


% \subsection{Remarques}



% \begin{itemize}
%     \item 
% \end{itemize}









% \subsection{Questions}



% \begin{itemize}
%     \item 
% \end{itemize}


% \subsection{Remarques}



% \begin{itemize}
%     \item 
% \end{itemize}









% \subsection{Questions}



% \begin{itemize}
%     \item 
% \end{itemize}


% \subsection{Remarques}



% \begin{itemize}
%     \item 
% \end{itemize}









% \subsection{Questions}



% \begin{itemize}
%     \item 
% \end{itemize}


% \subsection{Remarques}



% \begin{itemize}
%     \item 
% \end{itemize}









% \subsection{Questions}



% \begin{itemize}
%     \item 
% \end{itemize}


% \subsection{Remarques}



% \begin{itemize}
%     \item 
% \end{itemize}









% \subsection{Questions}



% \begin{itemize}
%     \item 
% \end{itemize}


% \subsection{Remarques}



% \begin{itemize}
%     \item 
% \end{itemize}











% \subsection{Questions}



% \begin{itemize}
%     \item 
% \end{itemize}


% \subsection{Remarques}



% \begin{itemize}
%     \item 
% \end{itemize}









% \subsection{Questions}



% \begin{itemize}
%     \item 
% \end{itemize}


% \subsection{Remarques}



% \begin{itemize}
%     \item 
% \end{itemize}







% \subsection{Questions}



% \begin{itemize}
%     \item 
% \end{itemize}


% \subsection{Remarques}



% \begin{itemize}
%     \item 
% \end{itemize}






% \subsection{Questions}



% \begin{itemize}
%     \item 
% \end{itemize}


% \subsection{Remarques}



% \begin{itemize}
%     \item 
% \end{itemize}









% \subsection{Questions}



% \begin{itemize}
%     \item 
% \end{itemize}


% \subsection{Remarques}



% \begin{itemize}
%     \item 
% \end{itemize}









% \subsection{Questions}



% \begin{itemize}
%     \item 
% \end{itemize}


% \subsection{Remarques}



% \begin{itemize}
%     \item 
% \end{itemize}








% \subsection{Questions}



% \begin{itemize}
%     \item 
% \end{itemize}


% \subsection{Remarques}



% \begin{itemize}
%     \item 
% \end{itemize}









% \subsection{Questions}



% \begin{itemize}
%     \item 
% \end{itemize}


% \subsection{Remarques}



% \begin{itemize}
%     \item 
% \end{itemize}









% \subsection{Questions}



% \begin{itemize}
%     \item 
% \end{itemize}


% \subsection{Remarques}



% \begin{itemize}
%     \item 
% \end{itemize}









% \subsection{Questions}



% \begin{itemize}
%     \item 
% \end{itemize}


% \subsection{Remarques}



% \begin{itemize}
%     \item 
% \end{itemize}









% \subsection{Questions}



% \begin{itemize}
%     \item 
% \end{itemize}


% \subsection{Remarques}



% \begin{itemize}
%     \item 
% \end{itemize}









% \subsection{Questions}



% \begin{itemize}
%     \item 
% \end{itemize}


% \subsection{Remarques}



% \begin{itemize}
%     \item 
% \end{itemize}









% \subsection{Questions}



% \begin{itemize}
%     \item 
% \end{itemize}


% \subsection{Remarques}



% \begin{itemize}
%     \item 
% \end{itemize}











% \subsection{Questions}



% \begin{itemize}
%     \item 
% \end{itemize}


% \subsection{Remarques}



% \begin{itemize}
%     \item 
% \end{itemize}









% \subsection{Questions}



% \begin{itemize}
%     \item 
% \end{itemize}


% \subsection{Remarques}



% \begin{itemize}
%     \item 
% \end{itemize}







% \subsection{Questions}



% \begin{itemize}
%     \item 
% \end{itemize}


% \subsection{Remarques}



% \begin{itemize}
%     \item 
% \end{itemize}






% \subsection{Questions}



% \begin{itemize}
%     \item 
% \end{itemize}


% \subsection{Remarques}



% \begin{itemize}
%     \item 
% \end{itemize}









% \subsection{Questions}



% \begin{itemize}
%     \item 
% \end{itemize}


% \subsection{Remarques}



% \begin{itemize}
%     \item 
% \end{itemize}









% \subsection{Questions}



% \begin{itemize}
%     \item 
% \end{itemize}


% \subsection{Remarques}



% \begin{itemize}
%     \item 
% \end{itemize}








% \subsection{Questions}



% \begin{itemize}
%     \item 
% \end{itemize}


% \subsection{Remarques}



% \begin{itemize}
%     \item 
% \end{itemize}









% \subsection{Questions}



% \begin{itemize}
%     \item 
% \end{itemize}


% \subsection{Remarques}



% \begin{itemize}
%     \item 
% \end{itemize}









% \subsection{Questions}



% \begin{itemize}
%     \item 
% \end{itemize}


% \subsection{Remarques}



% \begin{itemize}
%     \item 
% \end{itemize}









% \subsection{Questions}



% \begin{itemize}
%     \item 
% \end{itemize}


% \subsection{Remarques}



% \begin{itemize}
%     \item 
% \end{itemize}









% \subsection{Questions}



% \begin{itemize}
%     \item 
% \end{itemize}


% \subsection{Remarques}



% \begin{itemize}
%     \item 
% \end{itemize}









% \subsection{Questions}



% \begin{itemize}
%     \item 
% \end{itemize}


% \subsection{Remarques}



% \begin{itemize}
%     \item 
% \end{itemize}









% \subsection{Questions}



% \begin{itemize}
%     \item 
% \end{itemize}


% \subsection{Remarques}



% \begin{itemize}
%     \item 
% \end{itemize}











% \subsection{Questions}



% \begin{itemize}
%     \item 
% \end{itemize}


% \subsection{Remarques}



% \begin{itemize}
%     \item 
% \end{itemize}









% \subsection{Questions}



% \begin{itemize}
%     \item 
% \end{itemize}


% \subsection{Remarques}



% \begin{itemize}
%     \item 
% \end{itemize}







% \subsection{Questions}



% \begin{itemize}
%     \item 
% \end{itemize}


% \subsection{Remarques}



% \begin{itemize}
%     \item 
% \end{itemize}






% \subsection{Questions}



% \begin{itemize}
%     \item 
% \end{itemize}


% \subsection{Remarques}



% \begin{itemize}
%     \item 
% \end{itemize}









% \subsection{Questions}



% \begin{itemize}
%     \item 
% \end{itemize}


% \subsection{Remarques}



% \begin{itemize}
%     \item 
% \end{itemize}









% \subsection{Questions}



% \begin{itemize}
%     \item 
% \end{itemize}


% \subsection{Remarques}



% \begin{itemize}
%     \item 
% \end{itemize}








% \subsection{Questions}



% \begin{itemize}
%     \item 
% \end{itemize}


% \subsection{Remarques}



% \begin{itemize}
%     \item 
% \end{itemize}









% \subsection{Questions}



% \begin{itemize}
%     \item 
% \end{itemize}


% \subsection{Remarques}



% \begin{itemize}
%     \item 
% \end{itemize}









% \subsection{Questions}



% \begin{itemize}
%     \item 
% \end{itemize}


% \subsection{Remarques}



% \begin{itemize}
%     \item 
% \end{itemize}









% \subsection{Questions}



% \begin{itemize}
%     \item 
% \end{itemize}


% \subsection{Remarques}



% \begin{itemize}
%     \item 
% \end{itemize}









% \subsection{Questions}



% \begin{itemize}
%     \item 
% \end{itemize}


% \subsection{Remarques}



% \begin{itemize}
%     \item 
% \end{itemize}









% \subsection{Questions}



% \begin{itemize}
%     \item 
% \end{itemize}


% \subsection{Remarques}



% \begin{itemize}
%     \item 
% \end{itemize}









% \subsection{Questions}



% \begin{itemize}
%     \item 
% \end{itemize}


% \subsection{Remarques}



% \begin{itemize}
%     \item 
% \end{itemize}











% \subsection{Questions}



% \begin{itemize}
%     \item 
% \end{itemize}


% \subsection{Remarques}



% \begin{itemize}
%     \item 
% \end{itemize}














\bibliography{biblio}
\bibliographystyle{unsrt}




\end{document}