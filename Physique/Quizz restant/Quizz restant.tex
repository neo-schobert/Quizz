\documentclass[a4paper, 11pt, hidelinks]{article}
\usepackage{bookmark}
\usepackage[utf8]{inputenc} 
\usepackage[T1]{fontenc}
\usepackage{lmodern}
\usepackage{graphicx}
\usepackage[french]{babel}
\usepackage{geometry}
\usepackage{eucal}
\usepackage{caption}
\usepackage{float}
\usepackage{url}
\usepackage{amsmath}
\usepackage{amssymb}
\usepackage{color}
\usepackage{hyperref}
\usepackage{cancel}
\usepackage{tikz}
\usepackage{mathrsfs}  
\usepackage{esvect}


\geometry{hmargin=2cm,vmargin=1.5cm}

\tikzset{
  treenode/.style = {shape=rectangle, rounded corners,
                     draw, align=center,
                     top color=white, bottom color=blue!5},
  root/.style     = {treenode, font=\Large, bottom color=red!10},
  env/.style      = {treenode, font=\ttfamily\normalsize},
  dummy/.style    = {circle,draw}
}

\newcommand{\prp}{\large \textbf{Proposition :} \large}

\newcommand{\tm}{\large \textbf{Théoreme :} \large}

\newcommand{\ex}{\textcolor{green}{Exemple :} }

\newcommand{\dm}{\textcolor{red}{\textbf{Démo :} } }

\newcommand{\de}{\large \textbf{Définition} \large }

\newcommand{\rmq}{\textbf{Remarque :} }

\newcommand{\bs}{\bigskip}

\newcommand{\voca}{\textcolor{blue}{\textbf{Vocabulaire} } }

\newcommand{\lem}{\textcolor{red}{\textbf{Lemme :} } }

\newcommand{\trinom}[3]{\begin{pmatrix}
    #1 \\
    #2 \\
    #3
\end{pmatrix}}

\newcommand{\quadrinom}[4]{\begin{pmatrix}
    #1 \\
    #2 \\
    #3 \\
    #4 \\
\end{pmatrix}}

\newcommand{\pentanom}[5]{\begin{pmatrix}
    #1 \\
    #2 \\
    #3 \\
    #4 \\
    #5
\end{pmatrix}}

\newcommand{\hexanom}[6]{\begin{pmatrix}
    #1 \\
    #2 \\
    #3 \\
    #4 \\
    #5 \\
    #6 
\end{pmatrix}}

\newcommand{\serie}[2]{\displaystyle\sum_{#1 =0}^{+\infty} #2_{#1} }

\newcommand{\tend}{\underset{n \to + \infty}{\longrightarrow} }

\newcommand{\Lra}{\Leftrightarrow}

\newcommand{\lra}{\leftrightarrow}

\newcommand{\Ra}{\Rightarrow}

\newcommand{\ra}{\rightarrow}

\newcommand{\la}{\leftarrow}

\newcommand{\La}{\Leftarrow}

\newcommand{\dsum}[2]{\displaystyle\sum_{#1}^{#2} }

\newcommand{\dint}[2]{\displaystyle\int_{#1}^{#2} }

\newcommand{\ntend}{\underset{n \to + \infty}{\not \longrightarrow} }

\newenvironment{lmatrix}{$ \left|\begin{array}{l} }{\end{array}\right.$}

\newcommand{\img}[4]{\begin{figure}[!ht]
    \centering
    \includegraphics[scale=#1 ]{#2}
    \caption{#3}
    \label{#4}
    \end{figure} }    
\begin{document}

\newcommand{\grad}[1]{\vv{grad}#1}


\title{Physique 18-10}
\author{Schobert Néo}

\maketitle

\tableofcontents

\newpage 


\section{Ensemble des chapitres :}
\cite{Chapitre1}
\cite{Chapitre2}
\cite{Chapitre3}
\cite{Chapitre4}
\cite{Chapitre5}
\cite{Chapitre6}
\cite{Chapitre7}
\cite{Chapitre8}
\cite{Chapitre9}
\cite{Chapitre10}
\cite{Chapitre11}
\cite{Chapitre12}
\cite{Chapitre13}
\cite{Chapitre14}
\cite{Chapitre15}
\cite{Chapitre16}
\cite{Chapitre17}
\cite{Chapitre18}
\cite{Chapitre19}
\cite{Chapitre20}
\cite{Chapitre21}


\cite{Chapitre1bis}
\cite{Chapitre2bis}
\cite{Chapitre3bis}
\cite{Chapitre4bis}
\cite{Chapitre5bis}
\cite{Chapitre6bis}
\cite{Chapitre7bis}
\cite{Chapitre8bis}
\cite{Chapitre9bis}
\cite{Chapitre10bis}


\section{Questions restantes}
\begin{enumerate}
    \item Retrouver l'EDA 1D (corde) \cite{Chapitre15}
    \item Retrouver l'EDA 1D (Câble coaxial) \cite{Chapitre15}
    \item Quelles sont les variables "bonnes sa mère". Et pourquoi elles sont trop bonnes. \cite{Chapitre15}
    \item Retrouver l'EDA 1D avec les bonnes variables. \cite{Chapitre15}
    \item Définir polarisation rectiligne et circulaire. \cite{Chapitre15}
    \item Faire l'énergétique d'une OPPH. \cite{Chapitre15}
    \item Retrouver la vitesse de transport de l'énergie d'un OEM. \cite{Chapitre15}
    \item Comment calculer l'énergie d'un un volume élémentaire. ($2$ façons.) \cite{Chapitre15}
    \item Valeur moyenne en complexe. \cite{Chapitre15}
    \item Polarisation par dichroïsme. \cite{Chapitre15}
    \item Retrouver la loi de Malus. \cite{Chapitre15}
    \item Que peut-on dire sur le plasma (fréquence) \cite{Chapitre16}
    \item Définir un plasma \cite{Chapitre16}
    \item Quelles sont les hypothèses retenues ici ? \cite{Chapitre16}
    \item Calculer le rapport entre $\vv{f}_{magn}$ et $\vv{f}_{el}$ \cite{Chapitre16}
    \item Quelles autres forces considérer ? \cite{Chapitre16}
    \item Pourquoi c'est le même $\tau$ ? \cite{Chapitre16}
    \item Appliquer le RFD et retrouver $\vv{J}$, puis par loi d'Ohm locale, retrouver $\underline{\gamma}$ la conductivité complexe du plasma. \cite{Chapitre16}
    \item Que dire dans le cas où le gaz est plusieurs fois ionisé ? \cite{Chapitre16}
    \item Quelles sont les hypothèses pour un plasma dilué ? \cite{Chapitre16}
    \item Pourquoi ces hypothèses ? \cite{Chapitre16}
    \item En déduire la conductivité complexe simplifiée et le formalisme réel de $\vv{J}$ \cite{Chapitre16}
    \item Ecrire la conservation de la charge puis en déduire une pulsation de plasma. Que peut-on en déduire selon les cas $\omega=\omega_p$ et $\omega\neq \omega_p$. \cite{Chapitre16}
    \item Comment découpler les équations de Maxwell ? \cite{Chapitre16}
    \item Retrouver les équations de Maxwell complexe. \cite{Chapitre16}
    \item Quelle équation est modifié par rapport à l'OPPH classique ? \cite{Chapitre16}
    \item Comment faire l'analogie avec le cas du vide ? \cite{Chapitre16}
    \item Qu'est-ce que la relation de dispersion. \cite{Chapitre16}
    \item Comment l'établir dans le cas du plasma ? $2$ façons. \cite{Chapitre16}
    \item Qu'est-ce que la relation de Klein-Gordon. (relation de dispersion du plasma) \cite{Chapitre16}
    \item Que peut-on dire de la relation de dispersion du plasma ? \cite{Chapitre16}
    \item Retrouver $v_{\varphi}$ dans le cas $\omega > \omega_p$. \cite{Chapitre16}
    \item Pourquoi $v_{\varphi}>c$ ne pose pas de problème ? \cite{Chapitre16}
    \item Qu'est-ce que le domaine fréquentiel de transparence du plasma ? \cite{Chapitre16}
    \item Pourquoi le milieu du plasma est dispersif ? \cite{Chapitre16}
    \item Définir l'indice optique. \cite{Chapitre16}
    \item Qu'est-ce que le terme d'atténuation, comment le retrouver ? \cite{Chapitre16}
    \item Qu'est-ce que le domaine fréquentiel d'opacité ? \cite{Chapitre16}
    \item Définir la profondeur caractéristique de pénétration de l'onde dans le plasma. \cite{Chapitre16}
    \item Définir la notion d'onde Eva naissante. \cite{Chapitre16}
    \item Définir l'indice d'extinction. \cite{Chapitre16}
    \item Que peut-on dire du plasma ? \cite{Chapitre16}
    \item Donner la structure de l'OEM dans les cas $\omega > \omega_p$ et $\omega=\omega_p$. \cite{Chapitre16}
    \item Rappeler l'exemple de l'échangeur thermique. \cite{Chapitre19}
    \item Que représente-t-on dans un diagramme $(P,H)$ diphasé. \cite{Chapitre19}
    \item Représenter chacune des courbes dans un diagramme $(P,H)$ diphasé. \cite{Chapitre19}
    \item Rappeler le théorème du moment. Le retrouver.\footnote{$H_X=H_\ell + H_g$ puis $mh_X=m_\ell h_\ell + m_g h_g$ donc $h_X=(1-x_g)h_\ell + x_g h_g$ Finalement, $x_g = \frac{h_X-h_\ell}{h_g-h_\ell}$} \cite{Chapitre19}
    \item Définir l'équilibre physicochimique. \cite{Chapitre1bis}
    \item Condition de l'équilibre mécanique. \cite{Chapitre1bis}
    \item Condition de l'équilibre thermique. \cite{Chapitre1bis}
    \item Définir l'équilibre osmotique. \cite{Chapitre1bis}
    \item Donner les trois paramètres intensifs possibles en fonction de l'équilibre considéré. \cite{Chapitre1bis}
    \item Quel est le jeu naturel des variables extensives de $U$ cas des systèmes physiques ? et pourquoi \cite{Chapitre1bis}
    \item Quel est le jeu naturel des variables extensives de $U$ cas des systèmes physicochimiques ? et pourquoi \cite{Chapitre1bis}
    \item Calculer la différentielle de $U$ dans le cas des systèmes physicochimiques. \cite{Chapitre1bis}
    \item Définir alors le potentiel chimique puis la pression thermodynamique et la température thermodynamique \cite{Chapitre1bis}
    \item Cas du système physique non fermé (petite appartée) \cite{Chapitre1bis}
    \item Faire de même avec l'entropie. Définir de même chaque truc. \cite{Chapitre1bis}
    \item Que dire du sens d'évolution (vers quel équilibre) quand l'une des variables varie. \cite{Chapitre1bis}
    \item Qu'est-ce que l'expérience de Hertz ? \cite{Chapitre18}
    \item Définir un dipôle oscillant. \cite{Chapitre18}
    \item D'où vient la variation du moment dipolaire ? \cite{Chapitre18}
    \item Moment dipolaire oscillant d'un nuage électronique. Le retrouver \cite{Chapitre18}
    \item Moment dipolaire oscillant d'une antenne. \cite{Chapitre18}
    \item Rappeler les conditions de rayonnement. \cite{Chapitre18}
    \item Définir les trois échelles de longueur pertinentes. \cite{Chapitre18}
    \item Définir l'approximation dipolaire. \cite{Chapitre18}
    \item Définir l'approximation non relativiste. \cite{Chapitre18}
    \item Définir l'hypothèse de la zone de rayonnement. \cite{Chapitre18}
    \item Dans le cas du dipole oscillant, dans quelles approximations est-on ? \cite{Chapitre18}
    \item Expression du temps de retard. \cite{Chapitre18}
    \item Ecriture du temps de retard dans le cas d'une distribution plus étendue. \cite{Chapitre18}
    \item Définition de anistropie, cas de $\vv{B}$. \cite{Chapitre18}
    \item Que peut-on dire du dipole oscillant concernant l'énergie sur son axe. \cite{Chapitre18}
    \item Expression du champ électrique et du champ magnétique dans le cas du dipole oscillant en tout point. \cite{Chapitre18}
    \item Donner les trois cas auquel on peut être confronté dans le cas d'un dipole oscillant. \cite{Chapitre18}
    \item Valeur du champ magnétique et du champ électrique rayonné à grande distance par un dipôle oscillant. \cite{Chapitre18}
    \item Rappeler ici la structure d'onde plane de l'onde rayonnée. \cite{Chapitre18}
    \item Qu'est-ce que l'indicatrice de rayonnement ? \cite{Chapitre18}
    \item Comment calculer la puissance totale. \cite{Chapitre18}
    \item Donner la formule de Larmor, la retrouver. \cite{Chapitre18}
    \item Réutiliser le modèle de l'électron élastiquement lié pour retrouver le moment dipolaire. \cite{Chapitre18}
    \item Mener ensuite l'étude de la puissance rayonnée. \cite{Chapitre18}
    \item Qu'est-ce que la diffusion de Rayleigh, de Thompson ? \cite{Chapitre18}
    \item Comment en déduire que le ciel est bleu ? \cite{Chapitre18}
    \item Retrouver l'équilibre thermique et l'équilibre mécanique et l'équilibre osmotique à l'aide de l'entropie d'un système isolé $\Sigma_1 + \Sigma_2$. \cite{Chapitre1bis}
    \item Problème de l'entropie comme fonction d'état caractérisant le potentiel. \cite{Chapitre1bis}
    \item Décrire le phénomène de convection naturelle. \cite{Chapitre20}
    \item Décrire le phénomène de convection forcée. \cite{Chapitre20}
    \item Décrire le phénomène de rayonnement. \cite{Chapitre20}
    \item Que dire du rayonnement de tout corps de $T>0$\footnote{$u_{em}=\frac{8\pi h c }{\lambda^5} \frac{1}{e^{\frac{hc}{\lambda k_b T}}-1}$} \cite{Chapitre20}
    \item Comment que ca marche le corps noir déjà. \cite{Chapitre20}
    \item Que retenir du transfert d'énergie par rayonnement ? \footnote{Sans contact, sans matière, par OEM} \cite{Chapitre20}
    \item Décrire l'expérience d'Ingen Ousz. Quel résultat implique-t-elle ? \cite{Chapitre20} 
    \item Définir le flux thermique. (courant thermique) Et donner son unité. \cite{Chapitre20}
    \item Définir la densité de flux thermique surfacique. \cite{Chapitre20}
    \item Lien entre flux thermique et flux thermique surfacique. \cite{Chapitre20}
    \item Quelle propriété possède la densité de flux thermique surfacique ? \cite{Chapitre20}
    \item Définir le vecteur densité volumique de flux thermique. \cite{Chapitre20}
    \item Quelle propriété se propage au vecteur densité volumique de flux thermique ? \cite{Chapitre20}
    \item Quelle condition doit-on avoir pour définir la température habituellement. Comment la définir ? \cite{Chapitre20}
    \item Quelle problème cette définition pose-t-elle ? \cite{Chapitre20}
    \item Définir alors la température dans le cas de la conduciton thermique. \cite{Chapitre20}
    \item Définir les conditions de validité. \cite{Chapitre20}
    \item Donner la loi de Fourrier. Sur quoi s'appuie-t-elle ? Donner ses conditions d'application \cite{Chapitre20}
    \item Définir la conductivité thermique et donner son unité. \cite{Chapitre20}
    \item Ordre de grandeur de quelques matériaux. \cite{Chapitre20}
    \item Loi de Fourrier cas unidimensionnel. \cite{Chapitre20}
    \item Définir la capacité thermique élémentaire d'un système. \cite{Chapitre20}
    \item Quelle remarque peut-on faire dans le cas d'un milieu condensé. \cite{Chapitre20}
    \item Donner l'expression du premier principe dans le cas général. La retrouver. \cite{Chapitre20}
    \item Trouver une expression de $\mathscr{P}_{autre}$ \cite{Chapitre20}
    \item Trouver une expression de $dI_Q$ dans le cas $1D$. \cite{Chapitre20}
    \item Donner l'équation de la diffusion de la chaleur $1D$ (à l'aide du premier principe et de la loi de Fourrier) \cite{Chapitre20}
    \item Définir le coefficient de diffusion thermique. \cite{Chapitre20}
    \item Rappeler l'effet de peau. \cite{Chapitre20}
    \item Réécrire $dI_Q$ en géométrie cylindre. \cite{Chapitre20}
    \item Donner aussi $DI_Q$ en géométrie sphérique. \cite{Chapitre20}
    \item Retrouver alors l'équation de la diffusion généralisée  $3D$. \cite{Chapitre20}
    \item Donner les propriétés de l'équation de diffusion thermique. \cite{Chapitre20}
    \item Quelles sont les conditions aux limites usuelles ? \cite{Chapitre20}
    \item Comment trouver le temps caractéristique $\tau_c$. \cite{Chapitre20}
    \item Définir le nombre de Fourrier. \cite{Chapitre20}
    \item Résolution de l'équation de la chaleur cas régime permanent $1D$. \cite{Chapitre20}
    \item Définir ARQS thermique.\cite{Chapitre20}
    \item Conditions pour définir la résistance et conductance. \cite{Chapitre20}
    \item Définir alors la résistance et conductance. \cite{Chapitre20}
    \item Retrouver la résistance dans le cas d'une géométrie cartésienne / cylindrique / sphérique. \cite{Chapitre20}
    \item Retrouver les lois s'association. \cite{Chapitre20}
    \item Analogie complète entre thermique et électrique. \cite{Chapitre20}
    \item Rappeler la loi de Newton. Refaire le schéma. \cite{Chapitre20}
    \item Donner alors la loi de convection de Newton. \cite{Chapitre20}
    \item Définir alors le coefficient de transfert convectif. Donner son unité. \cite{Chapitre20}
    \item Définir alors la résistance conducto-convective. \cite{Chapitre20}
    \item Définir le nombre de Biot. A quoi il sert ? \cite{Chapitre20}
    \item Refaire les exos d'application. \cite{Chapitre20}
    \item Que déduit-t-on de l'expérience de Davisson et Germer ? \cite{Chapitre21}
    \item Donner le premier postulat de la physique quantique. \cite{Chapitre21}
    \item Rappeler la relation de Quantum de Planck. \cite{Chapitre21}
    \item Rappeler la relation de De Broglie. La retrouver. \cite{Chapitre21}
    \item Définir le critère quantique. Sens physique de la constante de Planck. \cite{Chapitre21}
    \item Définir la densité de probabilité de présence. \cite{Chapitre21}
    \item Définir la condition de normalisation. \cite{Chapitre21}
    \item Que permet la condition de normalisation ? \cite{Chapitre21}
    \item Définir la valeur moyenne d'un observable. \cite{Chapitre21}
    \item Donner la relation de Planck-Einstein. \cite{Chapitre21}
    \item Retrouver l'équation de Schrödinger 1D pour une particule libre. \cite{Chapitre21}
    \item Retrouver l'équation de Schrödinger 1D généralisée. Donner la forme 3D. \cite{Chapitre21}
    \item Que peut-on dire de l'équation de Schrödinger ? \cite{Chapitre21}
    \item Développer la densité de probabilité dans le cas des fentes d'Young. \cite{Chapitre21}
    \item Définir les états stationnaires. \cite{Chapitre21}
    \item Retrouver l'équation de Schrödinger indépendante du temps. Et son homologue indépendante de l'espace. \cite{Chapitre21}
    \item Donner la signification de $E$. \cite{Chapitre21}
    \item Comment s'écrit la fonction d'onde d'un état stationnaire ? \cite{Chapitre21}
    \item Que peut-on dire de la densité de probabilité dans le cas stationnaire ? \cite{Chapitre21}
    \item Différence onde sationnaire d'un point de vue classique ou quantique. \cite{Chapitre21}
    \item Donner le postulat numéro 3. \cite{Chapitre21}
    \item Conséquence de ce postulat. \cite{Chapitre21}
    \item Que peut-on dire des ondes de De Broglie, analogues aux OPPH. \cite{Chapitre21}
    \item Que peut-on dire de la fonction d'onde d'un système quantique réel (c'est-à-dire normalisable) ? \cite{Chapitre21}
    \item Donner la relation de dispersion de la particule libre. La retrouver \cite{Chapitre21}
    \item Rappeler la relation temps fréquence. \cite{Chapitre21}
    \item Retrouver alors le principe d'incertitude de Heisenberg. \cite{Chapitre21}
    \item Faire l'analogie avec l'EM et construire le vecteur courant de probabilité et l'équation de conservation de probabilité. \cite{Chapitre21} 
\end{enumerate}















\bibliography{biblio}
\bibliographystyle{unsrt}

\end{document}