\documentclass[a4paper, 11pt, hidelinks]{article}
\usepackage{bookmark}
\usepackage[utf8]{inputenc} 
\usepackage[T1]{fontenc}
\usepackage{lmodern}
\usepackage{graphicx}
\usepackage[french]{babel}
\usepackage{geometry}
\usepackage{eucal}
\usepackage{caption}
\usepackage{float}
\usepackage{url}
\usepackage{amsmath}
\usepackage{amssymb}
\usepackage{color}
\usepackage{hyperref}
\usepackage{cancel}
\usepackage{tikz}
\usepackage{mathrsfs}  
\usepackage{esvect}


\geometry{hmargin=2cm,vmargin=1.5cm}

\tikzset{
  treenode/.style = {shape=rectangle, rounded corners,
                     draw, align=center,
                     top color=white, bottom color=blue!5},
  root/.style     = {treenode, font=\Large, bottom color=red!10},
  env/.style      = {treenode, font=\ttfamily\normalsize},
  dummy/.style    = {circle,draw}
}

\newcommand{\prp}{\large \textbf{Proposition :} \large}

\newcommand{\tm}{\large \textbf{Théoreme :} \large}

\newcommand{\ex}{\textcolor{green}{Exemple :} }

\newcommand{\dm}{\textcolor{red}{\textbf{Démo :} } }

\newcommand{\de}{\large \textbf{Définition} \large }

\newcommand{\rmq}{\textbf{Remarque :} }

\newcommand{\bs}{\bigskip}

\newcommand{\voca}{\textcolor{blue}{\textbf{Vocabulaire} } }

\newcommand{\lem}{\textcolor{red}{\textbf{Lemme :} } }

\newcommand{\trinom}[3]{\begin{pmatrix}
    #1 \\
    #2 \\
    #3
\end{pmatrix}}

\newcommand{\quadrinom}[4]{\begin{pmatrix}
    #1 \\
    #2 \\
    #3 \\
    #4 \\
\end{pmatrix}}

\newcommand{\pentanom}[5]{\begin{pmatrix}
    #1 \\
    #2 \\
    #3 \\
    #4 \\
    #5
\end{pmatrix}}

\newcommand{\hexanom}[6]{\begin{pmatrix}
    #1 \\
    #2 \\
    #3 \\
    #4 \\
    #5 \\
    #6 
\end{pmatrix}}

\newcommand{\serie}[2]{\displaystyle\sum_{#1 =0}^{+\infty} #2_{#1} }

\newcommand{\tend}{\underset{n \to + \infty}{\longrightarrow} }

\newcommand{\Lra}{\Leftrightarrow}

\newcommand{\lra}{\leftrightarrow}

\newcommand{\Ra}{\Rightarrow}

\newcommand{\ra}{\rightarrow}

\newcommand{\la}{\leftarrow}

\newcommand{\La}{\Leftarrow}

\newcommand{\dsum}[2]{\displaystyle\sum_{#1}^{#2} }

\newcommand{\dint}[2]{\displaystyle\int_{#1}^{#2} }

\newcommand{\ntend}{\underset{n \to + \infty}{\not \longrightarrow} }

\newenvironment{lmatrix}{$ \left|\begin{array}{l} }{\end{array}\right.$}

\newcommand{\img}[4]{\begin{figure}[!ht]
    \centering
    \includegraphics[scale=#1 ]{#2}
    \caption{#3}
    \label{#4}
    \end{figure} }    
\begin{document}

\newcommand{\grad}[1]{\vv{grad}#1}


\title{Physique 18-10}
\author{Schobert Néo}

\maketitle

\tableofcontents

\newpage 


\section{Ensemble des chapitres :}
\cite{Chapitre1}
\cite{Chapitre2}
\cite{Chapitre3}
\cite{Chapitre4}
\cite{Chapitre5}
\cite{Chapitre6}
\cite{Chapitre7}
\cite{Chapitre8}
\cite{Chapitre9}
\cite{Chapitre10}
\cite{Chapitre11}
\cite{Chapitre12}
\cite{Chapitre13}
\cite{Chapitre14}
\cite{Chapitre15}
\cite{Chapitre16}
\cite{Chapitre17}
\cite{Chapitre18}
\cite{Chapitre19}
\cite{Chapitre20}
\cite{Chapitre21}


\cite{Chapitre1bis}
\cite{Chapitre2bis}
\cite{Chapitre3bis}
\cite{Chapitre4bis}
\cite{Chapitre5bis}
\cite{Chapitre6bis}
\cite{Chapitre7bis}
\cite{Chapitre8bis}
\cite{Chapitre9bis}
\cite{Chapitre10bis}


\section{Questions restantes}
\begin{enumerate}
    \item Calculer le moment et la résultante des actions subies par un dipole plongé dans un champ électrostatique uniforme. \cite{Chapitre12}
    \item Calculer le moment et la résultante des actions subies par un dipole plongé dans un champ électrostatique non uniforme. \cite{Chapitre12}
    \item Définition du moment magnétique. \cite{Chapitre12}
    \item unité du moment magnétique. \cite{Chapitre12}
    \item Moment cinétique électronique. \cite{Chapitre12}
    \item Moment dipolaire électronique. \cite{Chapitre12}
    \item Rapport gyromagnétique de l'électron. \cite{Chapitre12}
    \item Idée de moment de spin. \cite{Chapitre12}
    \item Définition du magnéton de Bohr. \cite{Chapitre12}
    \item Ordre de grandeurs de moments magnétiques. \cite{Chapitre12}
    \item Analogie entre électrique et magnétique. \cite{Chapitre12}
    \item Retrouver l'équation des lignes de champs. \cite{Chapitre12}
    \item Retrouver la valeur des actions mécaniques subiées par un dipôle magnétique plongé dans un champ magnétique extérieur uniforme. \cite{Chapitre12}
    \item Valeur des actions mécaniques subiées par un dipôle magnétique plongé dans un champ magnétique extérieur non uniforme \cite{Chapitre12}
    \item Notion du flux coupé. \cite{Chapitre12}
    \item Travail des forces de Laplace sur un circuit lors du déplacement $\vv{dr}$ \cite{Chapitre12}
    \item Théorème de Maxwell \cite{Chapitre12}
    \item Règle du flux maximal. \cite{Chapitre12}
    \item Equation de la conservation de la charge en $1D$. \cite{Chapitre13}
    \item Equation de la conservation de la charge en $3D$. \cite{Chapitre13}
    \item Retrouver la loi des noeuds en ARQS. \cite{Chapitre13}
    \item Retrouver l'équation de Maxwell-Ampère. \cite{Chapitre13}
    \item Que peut-on dire de l'intensité dans le condensateur. \cite{Chapitre13}
    \item Visualiser les effets du champ électrostatique sur le champ magnétique. (transport d'électricité) \cite{Chapitre13}
    \item Rappeler le phénomène d'induction. \cite{Chapitre13}
    \item Différence entre induction de Newmann et induction de Lorentz. \cite{Chapitre13}
    \item Retrouver la force électromotrice. \cite{Chapitre13}
    \item Rappeler la loi de Lenz-Faraday. \cite{Chapitre13}
    \item Retrouver l'équation de Maxwell-Faraday. \cite{Chapitre13}
    \item Donner les 4 équations de Maxwell en local et en global. \cite{Chapitre13}
    \item Valeur de la perméabilité du vide. \cite{Chapitre13}
    \item Valeur de la permittivité diélectrique du vide. \cite{Chapitre13}
    \item Lien entre $\mu_0$ et $\epsilon_0$ \cite{Chapitre13}
    \item Définition de l'ARQS. Ses critères de validité à redémontrer. \cite{Chapitre13}
    \item Bilan des équation de Maxwell en ARQS magnétique. \cite{Chapitre13}
    \item Définition de l'ARQS électrique. Ses caractères de validité à redémontrer. \cite{Chapitre13}
    \item Bilan des équations de Maxwell en ARQS électrique. \cite{Chapitre13}
    \item Quelles équations permettent de déduire que $\vv{E}$ et $\vv{B}$ sont couplés. \cite{Chapitre13}
    \item Quelles équations sont constitutives des champs $\vv{E}$ et $\vv{B}$ \cite{Chapitre13}
    \item Retrouver l'équation de d'Alembert pour le champ $\vv{E}$ et pour le champ $\vv{B}$ \cite{Chapitre13}
    \item Retrouver la loi D'Ohm locale. \cite{Chapitre14}
    \item Ordre de grandeurs de $\gamma$ \cite{Chapitre14}
    \item Retrouver la valeur de la résistance cas général et dans le cas d'un conducteur ohmique cylindrique de sections $S$ droite et de longueur $L$. \cite{Chapitre14}
    \item Retrouver la puissance cédée aux porteurs de charge. \cite{Chapitre14}
    \item Retrouver les 2 causes de variation de l'energie du champ électromagnétique. \cite{Chapitre14}
    \item Retrouver l'identité de Poynting. \cite{Chapitre14}
    \item Valeur du vecteur de Poynting. \cite{Chapitre14}
    \item Qu'est-ce que la densité volumique d'énergie électromagnétique / électrique / magnétique. \cite{Chapitre14}
    \item Théorème de Poynting. \cite{Chapitre14}
    \item Ordre de grandeur de flux surfaciques. \cite{Chapitre14}
    \item Ordre de grandeur $\frac{\epsilon_m}{\epsilon_e}$ \cite{Chapitre14}
    \item Retrouver l'EDA 1D (corde) \cite{Chapitre15}
    \item Retrouver l'EDA 1D (Câble coaxial) \cite{Chapitre15}
    \item Quelles sont les variables "bonnes sa mère". Et pourquoi elles sont trop bonnes. \cite{Chapitre15}
    \item Retrouver l'EDA 1D avec les bonnes variables. \cite{Chapitre15}
    \item Définir polarisation rectiligne et circulaire. \cite{Chapitre15}
    \item Faire l'énergétique d'une OPPH. \cite{Chapitre15}
    \item Retrouver la vitesse de transport de l'énergie d'un OEM. \cite{Chapitre15}
    \item Comment calculer l'énergie d'un un volume élémentaire. ($2$ façons.) \cite{Chapitre15}
    \item Valeur moyenne en complexe. \cite{Chapitre15}
    \item Polarisation par dichroïsme. \cite{Chapitre15}
    \item Retrouver la loi de Malus. \cite{Chapitre15}
    \item Que peut-on dire sur le plasma (fréquence) \cite{Chapitre16}
    \item Définir un plasma \cite{Chapitre16}
    \item Quelles sont les hypothèses retenues ici ? \cite{Chapitre16}
    \item Calculer le rapport entre $\vv{f}_{magn}$ et $\vv{f}_{el}$ \cite{Chapitre16}
    \item Quelles autres forces considérer ? \cite{Chapitre16}
    \item Pourquoi c'est le même $\tau$ ? \cite{Chapitre16}
    \item Appliquer le RFD et retrouver $\vv{J}$, puis par loi d'Ohm locale, retrouver $\underline{\gamma}$ la conductivité complexe du plasma. \cite{Chapitre16}
    \item Que dire dans le cas où le gaz est plusieurs fois ionisé ? \cite{Chapitre16}
    \item Quelles sont les hypothèses pour un plasma dilué ? \cite{Chapitre16}
    \item Pourquoi ces hypothèses ? \cite{Chapitre16}
    \item En déduire la conductivité complexe simplifiée et le formalisme réel de $\vv{J}$ \cite{Chapitre16}
    \item Ecrire la conservation de la charge puis en déduire une pulsation de plasma. Que peut-on en déduire selon les cas $\omega=\omega_p$ et $\omega\neq \omega_p$. \cite{Chapitre16}
    \item Comment découpler les équations de Maxwell ? \cite{Chapitre16}
    \item Retrouver les équations de Maxwell complexe. \cite{Chapitre16}
    \item Quelle équation est modifié par rapport à l'OPPH classique ? \cite{Chapitre16}
    \item Comment faire l'analogie avec le cas du vide ? \cite{Chapitre16}
    \item Qu'est-ce que la relation de dispersion. \cite{Chapitre16}
    \item Comment l'établir dans le cas du plasma ? $2$ façons. \cite{Chapitre16}
    \item Qu'est-ce que la relation de Klein-Gordon. (relation de dispersion du plasma) \cite{Chapitre16}
    \item Que peut-on dire de la relation de dispersion du plasma ? \cite{Chapitre16}
    \item Retrouver $v_{\varphi}$ dans le cas $\omega > \omega_p$. \cite{Chapitre16}
    \item Pourquoi $v_{\varphi}>c$ ne pose pas de problème ? \cite{Chapitre16}
    \item Qu'est-ce que le domaine fréquentiel de transparence du plasma ? \cite{Chapitre16}
    \item Pourquoi le milieu du plasma est dispersif ? \cite{Chapitre16}
    \item Définir l'indice optique. \cite{Chapitre16}
    \item Qu'est-ce que le terme d'atténuation, comment le retrouver ? \cite{Chapitre16}
    \item Qu'est-ce que le domaine fréquentiel d'opacité ? \cite{Chapitre16}
    \item Définir la profondeur caractéristique de pénétration de l'onde dans le plasma. \cite{Chapitre16}
    \item Définir la notion d'onde Eva naissante. \cite{Chapitre16}
    \item Définir l'indice d'extinction. \cite{Chapitre16}
    \item Que peut-on dire du plasma ? \cite{Chapitre16}
    \item Donner la structure de l'OEM dans les cas $\omega > \omega_p$ et $\omega=\omega_p$. \cite{Chapitre16}
    \item Rappeler l'exemple de l'échangeur thermique. \cite{Chapitre19}
    \item Que représente-t-on dans un diagramme $(P,H)$ diphasé. \cite{Chapitre19}
    \item Représenter chacune des courbes dans un diagramme $(P,H)$ diphasé. \cite{Chapitre19}
    \item Rappeler le théorème du moment. Le retrouver.\footnote{$H_X=H_\ell + H_g$ puis $mh_X=m_\ell h_\ell + m_g h_g$ donc $h_X=(1-x_g)h_\ell + x_g h_g$ Finalement, $x_g = \frac{h_X-h_\ell}{h_g-h_\ell}$} \cite{Chapitre19}
    \item Définir l'équilibre physicochimique. \cite{Chapitre1bis}
    \item Condition de l'équilibre mécanique. \cite{Chapitre1bis}
    \item Condition de l'équilibre thermique. \cite{Chapitre1bis}
    \item Définir l'équilibre osmotique. \cite{Chapitre1bis}
\end{enumerate}




\bibliography{biblio}
\bibliographystyle{unsrt}

\end{document}