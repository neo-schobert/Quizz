\documentclass[a4paper, 11pt, hidelinks]{article}
\usepackage{bookmark}
\usepackage[utf8]{inputenc} 
\usepackage[T1]{fontenc}
\usepackage{lmodern}
\usepackage{graphicx}
\usepackage[french]{babel}
\usepackage{geometry}
\usepackage{eucal}
\usepackage{caption}
\usepackage{float}
\usepackage{url}
\usepackage{amsmath}
\usepackage{amssymb}
\usepackage{color}
\usepackage{hyperref}
\usepackage{cancel}
\usepackage{tikz}
\usepackage{mathrsfs}  
\usepackage{esvect}


\geometry{hmargin=2cm,vmargin=1.5cm}

\tikzset{
  treenode/.style = {shape=rectangle, rounded corners,
                     draw, align=center,
                     top color=white, bottom color=blue!5},
  root/.style     = {treenode, font=\Large, bottom color=red!10},
  env/.style      = {treenode, font=\ttfamily\normalsize},
  dummy/.style    = {circle,draw}
}

\newcommand{\prp}{\large \textbf{Proposition :} \large}

\newcommand{\tm}{\large \textbf{Théoreme :} \large}

\newcommand{\ex}{\textcolor{green}{Exemple :} }

\newcommand{\dm}{\textcolor{red}{\textbf{Démo :} } }

\newcommand{\de}{\large \textbf{Définition} \large }

\newcommand{\rmq}{\textbf{Remarque :} }

\newcommand{\bs}{\bigskip}

\newcommand{\voca}{\textcolor{blue}{\textbf{Vocabulaire} } }

\newcommand{\lem}{\textcolor{red}{\textbf{Lemme :} } }

\newcommand{\trinom}[3]{\begin{pmatrix}
    #1 \\
    #2 \\
    #3
\end{pmatrix}}

\newcommand{\quadrinom}[4]{\begin{pmatrix}
    #1 \\
    #2 \\
    #3 \\
    #4 \\
\end{pmatrix}}

\newcommand{\pentanom}[5]{\begin{pmatrix}
    #1 \\
    #2 \\
    #3 \\
    #4 \\
    #5
\end{pmatrix}}

\newcommand{\hexanom}[6]{\begin{pmatrix}
    #1 \\
    #2 \\
    #3 \\
    #4 \\
    #5 \\
    #6 
\end{pmatrix}}

\newcommand{\serie}[2]{\displaystyle\sum_{#1 =0}^{+\infty} #2_{#1} }

\newcommand{\tend}{\underset{n \to + \infty}{\longrightarrow} }

\newcommand{\Lra}{\Leftrightarrow}

\newcommand{\lra}{\leftrightarrow}

\newcommand{\Ra}{\Rightarrow}

\newcommand{\ra}{\rightarrow}

\newcommand{\la}{\leftarrow}

\newcommand{\La}{\Leftarrow}

\newcommand{\dsum}[2]{\displaystyle\sum_{#1}^{#2} }

\newcommand{\dint}[2]{\displaystyle\int_{#1}^{#2} }

\newcommand{\ntend}{\underset{n \to + \infty}{\not \longrightarrow} }

\newenvironment{lmatrix}{$ \left|\begin{array}{l} }{\end{array}\right.$}

\newcommand{\img}[4]{\begin{figure}[!ht]
    \centering
    \includegraphics[scale=#1 ]{#2}
    \caption{#3}
    \label{#4}
    \end{figure} }    
\begin{document}

\newcommand{\grad}[1]{\vv{grad}#1}


\title{Physique 18-10}
\author{Schobert Néo}

\maketitle

\tableofcontents

\newpage 


\section{Ensemble des chapitres :}
\cite{Chapitre1}
\cite{Chapitre2}
\cite{Chapitre3}
\cite{Chapitre4}
\cite{Chapitre5}
\cite{Chapitre6}
\cite{Chapitre7}
\cite{Chapitre8}
\cite{Chapitre9}
\cite{Chapitre10}
\cite{Chapitre11}
\cite{Chapitre12}



\section{Questions restantes}
\begin{enumerate}
    \item Les charges doivent être au repos; qu'est-ce que ça implique sur le champ. \cite{Chapitre9}
    \item Définir la capacité d'un condensateur. \cite{Chapitre9}
    \item Valeur de la capacité d'un condensateur plan. \cite{Chapitre9}
    \item Valeur de la permittivité dans un milieu autre que le vide. \cite{Chapitre9}
    \item Calculer l'énergie électrique emmagasinée dans un condensateur. \cite{Chapitre9}
    \item Définir un tube de champ. \cite{Chapitre9}
    \item Propriété du champ par rapport aux zones isopotentielles. \cite{Chapitre9}
    \item Calculer le flux élémentaire du champ électrique à travers la surface fermée du méso-cube. \cite{Chapitre10}
    \item Définir la divergence. ($div(\vv{E}))$ \cite{Chapitre10}
    \item Qu'est ce que l'équation de Maxwell-Gauss. \cite{Chapitre10}
    \item Citer le théorème de Green-Ostrogradski. \cite{Chapitre10} 
    \item Calculer la circulation élémentaire du champ électrostatique sur le contour fermé. \cite{Chapitre10}
    \item Définir la rotationnelle. ($\vv{rot}\vv{E})$ \cite{Chapitre10}
    \item Equation de Maxwell-Faraday de la statique. \cite{Chapitre10}
    \item Autre expression de $\vv{rot}\vv{E}$. \cite{Chapitre10}
    \item Que remarque-t-on pour mémoriser plus facilement l'expression de $\vv{rot}\vv{E}$. \cite{Chapitre10}
    \item Citer le théorème de Stokes-Ampère. \cite{Chapitre10}
    \item Donner l'expression du courant. \cite{Chapitre11}
    \item Valeur du vecteur densité de courant dans le cas de plusieurs types de porteurs de charge. \cite{Chapitre11}
    \item Définition véritable du vecteur densité de courant. \cite{Chapitre11}
    \item Cas de la distribution surfacique. Et définition du vecteur densité surfacique de courant. \cite{Chapitre11}
    \item Donner la loi de Biot et Savart. $d\vv{B}_P(M)=\frac{\mu_0}{4 \pi}\frac{\vv{J}(P)\land \vv{PM}}{PM^3}d\tau$ \cite{Chapitre11}
    \item Définir le flux magnétostatique \cite{Chapitre11}
    \item Que peut-on dire du flux magnétostatique. \cite{Chapitre11}
    \item Quel est le lien avec la divergence de $\vv{B}$ \cite{Chapitre11}
    \item Equation de Maxwell-Thomson. \cite{Chapitre11}
    \item Valeur de $\vv{B}$ grâce à la loi de Biot Savart. \cite{Chapitre11}
    \item Définition de la circulation du champ magnétique. \cite{Chapitre11}
    \item Discussion en fonction de $\Gamma$ \cite{Chapitre11}
    \item Citer le théorème d'Ampère. \cite{Chapitre11}
    \item Valeur de la perméabilité magnétique du vide. \cite{Chapitre11}
    \item Que vaut $I_{enlace}$ dans le cas d'une distribution filiforme / volumique / surfacique. \cite{Chapitre11}
    \item Donner la stratégie de mise en \oe uvre. \cite{Chapitre11}
    \item Rappeler les conditions pour appliquer le théorème d'Ampère "idéal". \cite{Chapitre11}
    \item Que peut-on dire du champ magnétique ? \cite{Chapitre11}
    \item Comment faire pour utiliser Ampère dans le cas du solénoide infini / de la nappe de courant ? \cite{Chapitre11}
    \item Rappeler l'équation de Maxwell-Ampère et sa "preuve". \cite{Chapitre11}
    \item Autour de quoi tourne le courant magnétostatique ? \cite{Chapitre11}
    \item Que se passe-t-il pour le champ magnétostatique lors d'un évasement / resserrement. \cite{Chapitre11}
    \item Capacité linéique $C_{\ell}=\frac{C_H}{H}$
    \item Calculer un vecteur densité volumique de courant. (voir TD8 exo 2)
    \item Définir un dipole \cite{Chapitre12}
    \item Définir le moment dipolaire \cite{Chapitre12}
    \item Moment dipolaire dans le cas de $n$ charges. (Voir chapitre 12 Fiches)
    \item Définir le Debye. \cite{Chapitre12}
    \item Définition du barycentre. \cite{Chapitre12}
    \item Calcul du potentiel électrostatique en approximation dipolaire. \cite{Chapitre12}
    \item Valeur du potentiel électrostatique en approximation dipolaire. \cite{Chapitre12}
    \item Valeur du champ électrostatique dipolaire. \cite{Chapitre12}
    \item Calculer le champ électrostatique dipolaire. \cite{Chapitre12}
    \item Définir les positions de Gauss. \cite{Chapitre12}
    \item Trouver l'équation des lignes de champs. \cite{Chapitre12}
    \item Trouver l'équation des isopotentielles. \cite{Chapitre12}
    \item Calculer le moment et la résultante des actions subies par un dipole plongé dans un champ électrostatique uniforme. \cite{Chapitre12}
    \item Calculer le moment et la résultante des actions subies par un dipole plongé dans un champ électrostatique non uniforme. \cite{Chapitre12}
    \item Lien entre la force et l'énergie potentielle. \cite{Chapitre12}
    \item Inexistence du monopole magnétique. \cite{Chapitre12}
    \item On a que des dipoles. \cite{Chapitre12}
    \item Comment le montrer ? \cite{Chapitre12}
    \item Définition du moment magnétique. \cite{Chapitre12}
    \item unité du moment magnétique. \cite{Chapitre12}
    \item Moment cinétique électronique. \cite{Chapitre12}
    \item Moment dipolaire électronique. \cite{Chapitre12}
    \item Rapport gyromagnétique de l'électron. \cite{Chapitre12}
    \item Idée de moment de spin. \cite{Chapitre12}
    \item Définition du magnéton de Bohr. \cite{Chapitre12}
    \item Ordre de grandeurs de moments magnétiques. \cite{Chapitre12}
    \item Analogie entre électrique et magnétique. \cite{Chapitre12}
    \item Retrouver l'équation des lignes de champs. \cite{Chapitre12}
    \item Retrouver la valeur des actions mécaniques subiées par un dipôle magnétique plongé dans un champ magnétique extérieur uniforme. \cite{Chapitre12}
    \item Valeur des actions mécaniques subiées par un dipôle magnétique plongé dans un champ magnétique extérieur non uniforme \cite{Chapitre12}
    \item Notion du flux coupé. \cite{Chapitre12}
    \item Travail des forces de Laplace sur un circuit lors du déplacement $\vv{dr}$ \cite{Chapitre12}
    \item Théorème de Maxwell \cite{Chapitre12}
    \item Règle du flux maximal. \cite{Chapitre12}
    \item Equation de la conservation de la charge en $1D$. \cite{Chapitre13}
    \item Equation de la conservation de la charge en $3D$. \cite{Chapitre13}
    \item Retrouver la loi des noeuds en ARQS. \cite{Chapitre13}
    \item Retrouver l'équation de Maxwell-Ampère. \cite{Chapitre13}
    \item Que peut-on dire de l'intensité dans le condensateur. \cite{Chapitre13}
    \item Visualiser les effets du champ électrostatique sur le champ magnétique. (transport d'électricité) \cite{Chapitre13}
    \item Rappeler le phénomène d'induction. \cite{Chapitre13}
    \item Différence entre induction de Newmann et induction de Lorentz. \cite{Chapitre13}
    \item Retrouver la force électromotrice. \cite{Chapitre13}
    \item Rappeler la loi de Lenz-Faraday. \cite{Chapitre13}
    \item Retrouver l'équation de Maxwell-Faraday. \cite{Chapitre13}
    \item Donner les 4 équations de Maxwell en local et en global. \cite{Chapitre13}
    \item Valeur de la perméabilité du vide. \cite{Chapitre13}
    \item Valeur de la permittivité diélectrique du vide. \cite{Chapitre13}
    \item Lien entre $\mu_0$ et $\epsilon_0$ \cite{Chapitre13}
    \item Définition de l'ARQS. Ses critères de validité à redémontrer. \cite{Chapitre13}
    \item Bilan des équation de Maxwell en ARQS magnétique. \cite{Chapitre13}
    \item Définition de l'ARQS électrique. Ses caractères de validité à redémontrer. \cite{Chapitre13}
    \item Bilan des équations de Maxwell en ARQS électrique. \cite{Chapitre13}
    \item Quelles équations permettent de déduire que $\vv{E}$ et $\vv{B}$ sont couplés. \cite{Chapitre13}
    \item Quelles équations sont constitutives des champs $\vv{E}$ et $\vv{B}$ \cite{Chapitre13}
    \item Retrouver l'équation de d'Alembert pour le champ $\vv{E}$ et pour le champ $\vv{B}$ \cite{Chapitre13}
    \item Retrouver la loi D'Ohm locale. \cite{Chapitre14}
    \item Ordre de grandeurs de $\gamma$ \cite{Chapitre14}
    \item Retrouver la valeur de la résistance cas général et dans le cas d'un conducteur ohmique cylindrique de sections $S$ droite et de longueur $L$. \cite{Chapitre14}
    \item Retrouver la puissance cédée aux porteurs de charge. \cite{Chapitre14}
    \item Retrouver les 2 causes de variation de l'energie du champ électromagnétique. \cite{Chapitre14}
    \item Retrouver l'identité de Poynting. \cite{Chapitre14}
    \item Valeur du vecteur de Poynting. \cite{Chapitre14}
    \item Qu'est-ce que la densité volumique d'énergie électromagnétique / électrique / magnétique. \cite{Chapitre14}
    \item Théorème de Poynting. \cite{Chapitre14}
    \item Ordre de grandeur de flux surfaciques. \cite{Chapitre14}
    \item Ordre de grandeur $\frac{\epsilon_m}{\epsilon_e}$ \cite{Chapitre14}
    \item Retrouver l'EDA 1D (corde) \cite{Chapitre15}
    \item Retrouver l'EDA 1D (Câble coaxial) \cite{Chapitre15}
    \item Quelles sont les variables "bonnes sa mère". Et pourquoi elles sont trop bonnes. \cite{Chapitre15}
    \item Retrouver l'EDA 1D avec les bonnes variables. \cite{Chapitre15}
\end{enumerate}






\bibliography{biblio}
\bibliographystyle{unsrt}

\end{document}